-----------------------------------------------------------

	  -*- Mode: Text -*-   Notes on updating this file:

This file is maintained at three locations.   It is  AIWORD.RF[UP,DOC]
at SAIL, and GLS;JARGON >  at MIT-MC and at MIT-AI.   If you make  any
changes, please FTP the new file to the other location.	  (NOTE:   Use
ASCII mode in FTP to avoid screwing up the tilde char!)	  It is also a
good idea to compare this file	against the copy on the other  machine
before FTP'ing and to merge any	 changes found there,  in case someone
else forgot to do the FTP.    Also, please  let us know	 (see list  of
names below) about your changes so that we can double-check them.

Try to conform to the format already being used--70 character lines,
3-character indentations, pronunciations in parentheses, etymologies
in brackets, single-space after def'n numbers and word classes, etc.

Stick to the standard ASCII character set.

If you'd rather not mung the file yourself, send your definitions to
DON @ SAIL, GLS @ MIT-AI, and/or MRC @ SAIL.

The last edit (of this line, anyway) was by Don Woods, 82-11-14.

======================================================================

	Compiled by Guy L. Steele Jr., Raphael Finkel, Donald
	Woods,	Geoff  Goodfellow  and	Mark  Crispin,	 with
	assistance from the MIT	 and Stanford AI  communities
	and   Worcester	   Polytechnic	  Institute.	 Some
	contributions were  submitted  via the	ARPAnet	 from
	miscellaneous sites.

Verb doubling: a standard construction is to double a verb and use it
   as a comment on what the implied subject does.  Often used to
   terminate a conversation.  Typical examples involve WIN, LOSE,
   HACK, FLAME, BARF, CHOMP:
	"The disk heads just crashed."	"Lose, lose."
	"Mostly he just talked about his --- crock.  Flame, flame."
	"Boy, what a bagbiter!	Chomp, chomp!"

Soundalike slang: similar to Cockney rhyming slang.  Often made up on
   the spur of the moment.  Standard examples:
	Boston Globe => Boston Glob
	Herald American => Horrid (Harried) American
	New York Times => New York Slime
	historical reasons => hysterical raisins
	government property - do not duplicate (seen on keys)
		=> government duplicity - do not propagate
   Often the substitution will be made in such a way as to slip in
   a standard jargon word:
	Dr. Dobb's Journal => Dr. Frob's Journal
	creeping featurism => feeping creaturism
	Margaret Jacks Hall => Marginal Hacks Hall

The -P convention: turning a word into a question by appending the
   syllable "P"; from the LISP convention of appending the letter "P"
   to denote a predicate (a Boolean-values function).  The question
   should expect a yes/no answer, though it needn't.  (See T and NIL.)
     At dinnertime: "Foodp?"  "Yeah, I'm pretty hungry." or "T!"
     "State-of-the-world-P?"  (Straight) "I'm about to go home."
			      (Humorous) "Yes, the world has a state."
   [One of the best of these is a Gosperism (i.e., due to Bill
   Gosper).  When we were at a Chinese restaurant, he wanted to know
   whether someone would like to share with him a two-person-sized
   bowl of soup.  His inquiry was: "Split-p soup?" --GLS]

Peculiar nouns: MIT AI hackers love to take various words and add the
   wrong endings to them to make nouns and verbs, often by extending a
   standard rule to nonuniform cases.  Examples:
		porous => porosity
		generous => generosity
	Ergo:	mysterious => mysteriosity
		ferrous => ferocity
   Other examples:  winnitude, disgustitude, hackification.

Spoken inarticulations:	 Words such as "mumble", "sigh", and "groan"
   are spoken in places where their referent might more naturally be
   used.  It has been suggested that this usage derives from the
   impossibility of representing such noises in a com link.  Another
   expression sometimes heard is "complain!"

@BEGIN (primarily CMU) with @END, used humorously in writing to
   indicate a context or to remark on the surrounded text.  From the
   SCRIBE command of the same name.  For example:
	@Begin(Flame)
	Predicate logic is the only good programming language.
	Anyone who would use anything else is an idiot.	 Also,
	computers should be tredecimal instead of binary.
	@End(Flame)

ANGLE BRACKETS (primarily MIT) n. Either of the characters "<" and
   ">".	 See BROKET.

AOS (aus (East coast) ay-ahs (West coast)) [based on a PDP-10
   increment instruction] v. To increase the amount of something.
   "Aos the campfire."	Usage: considered silly.  See SOS.

ARG n. Abbreviation for "argument" (to a function), used so often as
   to have become a new word.

AUTOMAGICALLY adv. Automatically, but in a way which, for some reason
   (typically because it is too complicated, or too ugly, or perhaps
   even too trivial), I don't feel like explaining to you.  See MAGIC.
   Example: Some programs which produce XGP output files spool them
   automagically.

BAGBITER 1. n. Equipment or program that fails, usually
   intermittently.  2. BAGBITING: adj. Failing hardware or software.
   "This bagbiting system won't let me get out of spacewar."  Usage:
   verges on obscenity.	 Grammatically separable; one may speak of
   "biting the bag".  Synonyms: LOSER, LOSING, CRETINOUS, BLETCHEROUS,
   BARFUCIOUS, CHOMPER, CHOMPING.

BANG n. Common alternate name for EXCL (q.v.), especially at CMU.  See
   SHRIEK.

BAR 1. The second metasyntactic variable, after FOO.  "Suppose we have
   two functions FOO and BAR.  FOO calls BAR..."  2. Often appended to
   FOO to produce FOOBAR.

BARF [from the "layman" slang, meaning "vomit"] 1. interj. Term of
   disgust.  See BLETCH.  2. v. Choke, as on input.  May mean to give
   an error message.  "The function `=' compares two fixnums or two
   flonums, and barfs on anything else."  3. BARFULOUS, BARFUCIOUS:
   adj. Said of something which would make anyone barf, if only for
   aesthetic reasons.

BELLS AND WHISTLES n. Unnecessary but useful (or amusing) features of
   a program.  "Now that we've got the basic program working, let's go
   back and add some bells and whistles."  Nobody seems to know what
   distinguishes a bell from a whistle.

BIGNUMS [from Macsyma] n. 1. In backgammon, large numbers on the dice.
   2. Multiple-precision (sometimes infinitely extendable) integers
   and, through analogy, any very large numbers.  3. EL CAMINO BIGNUM:
   El Camino Real, a street through the San Francisco peninsula that
   originally extended (and still appears in places) all the way to
   Mexico City.	 It was termed "El Camino Double Precision" when
   someone noted it was a very long street, and then "El Camino
   Bignum" when it was pointed out that it was hundreds of miles long.

BIN [short for BINARY; used as a second file name on ITS] 1. n.
   BINARY.  2. BIN FILE: A file containing the BIN for a program.
   Usage: used at MIT, which runs on ITS.  The equivalent term at
   Stanford is DMP (pronounced "dump") FILE.  Other names used include
   SAV ("save") FILE (DEC and Tenex), SHR ("share") and LOW FILES
   (DEC), and EXE ("ex'ee") FILE (DEC and Twenex).  Also in this
   category are the input files to the various flavors of linking
   loaders (LOADER, LINK-10, STINK), called REL FILES.

BINARY n. The object code for a program.

BIT n. 1. The unit of information; the amount of information obtained
   by asking a yes-or-no question.  "Bits" is often used simply to
   mean information, as in "Give me bits about DPL replicators".  2.
   [By extension from "interrupt bits" on a computer] A reminder that
   something should be done or talked about eventually.	 Upon seeing
   someone that you haven't talked to for a while, it's common for one
   or both to say, "I have a bit set for you."

BITBLT (bit'blit) 1. v. To perform a complex operation on a large
   block of bits, usually involving the bits being displayed on a
   bitmapped raster screen.  See BLT.  2. n. The operation itself.

BIT BUCKET n. 1. A receptacle used to hold the runoff from the
   computer's shift registers.	2. Mythical destination of deleted
   files, GC'ed memory, and other no-longer-accessible data.  3. The
   physical device associated with "NUL:".

BLETCH [from German "brechen", to vomit (?)] 1. interj. Term of
   disgust.  2. BLETCHEROUS: adj. Disgusting in design or function.
   "This keyboard is bletcherous!"  Usage: slightly comic.

BLT (blit, very rarely belt) [based on the PDP-10 block transfer
   instruction; confusing to users of the PDP-11] 1. v. To transfer a
   large contiguous package of information from one place to another.
   2. THE BIG BLT: n. Shuffling operation on the PDP-10 under some
   operating systems that consumes a significant amount of computer
   time.  3. (usually pronounced B-L-T) n. Sandwich containing bacon,
   lettuce, and tomato.

BOGOSITY n. The degree to which something is BOGUS (q.v.).  At CMU,
   bogosity is measured with a bogometer; typical use: in a seminar,
   when a speaker says something bogus, a listener might raise his
   hand and say, "My bogometer just triggered."	 The agreed-upon unit
   of bogosity is the microLenat (uL).

BOGUS (WPI, Yale, Stanford) adj. 1. Non-functional.  "Your patches are
   bogus."  2. Useless.	 "OPCON is a bogus program."  3. False.	 "Your
   arguments are bogus."  4. Incorrect.	 "That algorithm is bogus."
   5. Silly.  "Stop writing those bogus sagas."	 (This word seems to
   have some, but not all, of the connotations of RANDOM.)
   [Etymological note from Lehman/Reid at CMU:	"Bogus" was originally
   used (in this sense) at Princeton, in the late 60's.	 It was used
   not particularly in the CS department, but all over campus.	It
   came to Yale, where one of us (Lehman) was an undergraduate, and
   (we assume) elsewhere through the efforts of Princeton alumni who
   brought the word with them from their alma mater.  In the Yale
   case, the alumnus is Michael Shamos, who was a graduate student at
   Yale and is now a faculty member here.  A glossary of bogus words
   was compiled at Yale when the word was first popularized (e.g.,
   autobogophobia: the fear of becoming bogotified).]

BOUNCE (Stanford) v. To play volleyball.  "Bounce, bounce!  Stop
   wasting time on the computer and get out to the court!"

BRAIN-DAMAGED [generalization of "Honeywell Brain Damage" (HBD), a
   theoretical disease invented to explain certain utter cretinisms in
   Multics] adj. Obviously wrong; cretinous; demented.	There is an
   implication that the person responsible must have suffered brain
   damage, because he should have known better.	 Calling something
   brain-damaged is really bad; it also implies it is unusable.

BREAK v. 1. To cause to be broken (in any sense).  "Your latest patch
   to the system broke the TELNET server."  2. (of a program) To stop
   temporarily, so that it may be examined for debugging purposes.
   The place where it stops is a BREAKPOINT.

BROKEN adj. 1. Not working properly (of programs).  2. Behaving
   strangely; especially (of people), exhibiting extreme depression.

BROKET [by analogy with "bracket": a "broken bracket"] (primarily
   Stanford) n. Either of the characters "<" and ">".  (At MIT, and
   apparently in The Real World (q.v.) as well, these are usually
   called ANGLE BRACKETS.)

BUCKY BITS (primarily Stanford) n. The bits produced by the CTRL and
   META shift keys on a Stanford (or Knight) keyboard.	Rumor has it
   that the idea for extra bits for characters came from Niklaus
   Wirth, and that his nickname was `Bucky'.
   DOUBLE BUCKY: adj. Using both the CTRL and META keys.  "The command
   to burn all LEDs is double bucky F."

BUG [from telephone terminology, "bugs in a telephone cable", blamed
   for noisy lines; however, Jean Sammet has repeatedly been heard to
   claim that the use of the term in CS comes from a story concerning
   actual bugs found wedged in an early malfunctioning computer] n. An
   unwanted and unintended property of a program.  (People can have
   bugs too (even winners) as in "PHW is a super winner, but he has
   some bugs.")	 See FEATURE.

BUM 1. v. To make highly efficient, either in time or space, often at
   the expense of clarity.  The object of the verb is usually what was
   removed ("I managed to bum three more instructions.") but can be
   the program being changed ("I bummed the inner loop down to seven
   microseconds.")  2. n. A small change to an algorithm to make it
   more efficient.

BUZZ v. To run in a very tight loop, perhaps without guarantee of
   getting out.

CANONICAL adj. The usual or standard state or manner of something.
   A true story:  One Bob Sjoberg, new at the MIT AI Lab, expressed
   some annoyance at the use of jargon.	 Over his loud objections, we
   made a point of using jargon as much as possible in his presence,
   and eventually it began to sink in.	Finally, in one conversation,
   he used the word "canonical" in jargon-like fashion without
   thinking.
   Steele: "Aha!  We've finally got you talking jargon too!"
   Stallman: "What did he say?"
   Steele: "He just used `canonical' in the canonical way."

CATATONIA (kat-uh-toe'nee-uh) n. A condition of suspended animation in
   which the system is in a wedged (CATATONIC) state.

CDR (ku'der) [from LISP] v. With "down", to trace down a list of
   elements.  "Shall we cdr down the agenda?"  Usage: silly.

CHINE NUAL n. The Lisp Machine Manual, so called because the title is
   wrapped around the cover so only those letters show.

CHOMP v. To lose; to chew on something of which more was bitten off
   than one can.  Probably related to gnashing of teeth.  See
   BAGBITER.  A hand gesture commonly accompanies this, consisting of
   the four fingers held together as if in a mitten or hand puppet,
   and the fingers and thumb open and close rapidly to illustrate a
   biting action.  The gesture alone means CHOMP CHOMP (see Verb
   Doubling).

CLOSE n. Abbreviation for "close (or right) parenthesis", used when
   necessary to eliminate oral ambiguity.  See OPEN.

COKEBOTTLE n. Any very unusual character.  MIT people complain about
   the "control-meta-cokebottle" commands at SAIL, and SAIL people
   complain about the "altmode-altmode-cokebottle" commands at MIT.

COM MODE (variant: COMM MODE) [from the ITS feature for linking two or
   more terminals together so that text typed on any is echoed on all,
   providing a means of conversation among hackers] n. The state a
   terminal is in when linked to another in this way.  Com mode has a
   special set of jargon words, used to save typing, which are not
   used orally:
	BCNU	Be seeing you.
	BTW	By the way...
	BYE?	Are you ready to unlink?  (This is the standard way to
		end a com mode conversation; the other person types
		BYE to confirm, or else continues the conversation.)
	CUL	See you later.
	FOO?	A greeting, also meaning R U THERE?  Often used in the
		case of unexpected links, meaning also "Sorry if I
		butted in" (linker) or "What's up?" (linkee).
	FYI	For your information...
	GA	Go ahead (used when two people have tried to type
		simultaneously; this cedes the right to type to
		the other).
	HELLOP	A greeting, also meaning R U THERE?  (An instance
		of the "-P" convention.)
	MtFBWY	May the Force be with you.  (From Star Wars.)
	NIL	No (see the main entry for NIL).
	OBTW	Oh, by the way...
	R U THERE?	Are you there?
	SEC	Wait a second (sometimes written SEC...).
	T	Yes (see the main entry for T).
	TNX	Thanks.
	TNX 1.0E6	Thanks a million (humorous).
	<double CRLF>  When the typing party has finished, he types
		two CRLF's to signal that he is done; this leaves a
		blank line between individual "speeches" in the
		conversation, making it easier to re-read the
		preceding text.
	<name>: When three or more terminals are linked, each speech
		is preceded by the typist's login name and a colon (or
		a hyphen) to indicate who is typing.  The login name
		often is shortened to a unique prefix (possibly a
		single letter) during a very long conversation.
	/\/\/\	The equivalent of a giggle.
   At Stanford, where the link feature is implemented by "talk loops",
   the term TALK MODE is used in place of COM MODE.  Most of the above
   "sub-jargon" is used at both Stanford and MIT.

CONNECTOR CONSPIRACY [probably came into prominence with the
   appearance of the KL-10, none of whose connectors match anything
   else] n. The tendency of manufacturers (or, by extension,
   programmers or purveyors of anything) to come up with new products
   which don't fit together with the old stuff, thereby making you buy
   either all new stuff or expensive interface devices.

CONS [from LISP] 1. v. To add a new element to a list.	2. CONS UP:
   v. To synthesize from smaller pieces: "to cons up an example".

CRASH 1. n. A sudden, usually drastic failure.	Most often said of the
   system (q.v., definition #1), sometimes of magnetic disk drives.
   "Three lusers lost their files in last night's disk crash."	A disk
   crash which entails the read/write heads dropping onto the surface
   of the disks and scraping off the oxide may also be referred to as
   a "head crash".  2. v. To fail suddenly.  "Has the system just
   crashed?"  Also used transitively to indicate the cause of the
   crash (usually a person or a program, or both).  "Those idiots
   playing spacewar crashed the system."  Sometimes said of people.
   See GRONK OUT.

CRETIN 1. n. Congenital loser (q.v.).  2. CRETINOUS: adj. See
   BLETCHEROUS and BAGBITING.  Usage: somewhat ad hominem.

CRLF (cur'lif, sometimes crul'lif) n. A carriage return (CR) followed
   by a line feed (LF).	 See TERPRI.

CROCK [probably from "layman" slang, which in turn may be derived from
   "crock of shit"] n. An awkward feature or programming technique
   that ought to be made cleaner.  Example: Using small integers to
   represent error codes without the program interpreting them to the
   user is a crock.  Also, a technique that works acceptably but which
   is quite prone to failure if disturbed in the least, for example
   depending on the machine opcodes having particular bit patterns so
   that you can use instructions as data words too; a tightly woven,
   almost completely unmodifiable structure.

CRUFTY [from "cruddy"] adj. 1. Poorly built, possibly overly complex.
   "This is standard old crufty DEC software".	Hence CRUFT, n. shoddy
   construction.  Also CRUFT, v. [from hand cruft, pun on hand craft]
   to write assembler code for something normally (and better) done by
   a compiler.	2. Unpleasant, especially to the touch, often with
   encrusted junk.  Like spilled coffee smeared with peanut butter and
   catsup.  Hence CRUFT, n. disgusting mess.  3. Generally unpleasant.
   CRUFTY or CRUFTIE n. A small crufty object (see FROB); often one
   which doesn't fit well into the scheme of things.  "A LISP property
   list is a good place to store crufties (or, random cruft)."
   [Note:  Does CRUFT have anything to do with the Cruft Lab at
   Harvard?  I don't know, though I was a Harvard student. - GLS]

CRUNCH v. 1. To process, usually in a time-consuming or complicated
   way.	 Connotes an essentially trivial operation which is
   nonetheless painful to perform.  The pain may be due to the
   triviality being imbedded in a loop from 1 to 1000000000.  "FORTRAN
   programs do mostly number crunching."  2. To reduce the size of a
   file by a complicated scheme that produces bit configurations
   completely unrelated to the original data, such as by a Huffman
   code.  (The file ends up looking like a paper document would if
   somebody crunched the paper into a wad.)  Since such compression
   usually takes more computations than simpler methods such as
   counting repeated characters (such as spaces) the term is doubly
   appropriate.	 (This meaning is usually used in the construction
   "file crunch(ing)" to distinguish it from "number crunch(ing)".)
   3. n. The character "#".  Usage: used at Xerox and CMU, among other
   places.  Other names for "#" include SHARP, NUMBER, HASH, PIG-PEN,
   POUND-SIGN, and MESH.  GLS adds: I recall reading somewhere that
   most of these are names for the # symbol IN CONTEXT.	 The name for
   the sign itself is "octothorp".

CTY (city) n. The terminal physically associated with a computer's
   operating console.

CUSPY [from the DEC acronym CUSP, for Commonly Used System Program,
   i.e., a utility program used by many people] (WPI) adj. 1. (of a
   program) Well-written.  2. Functionally excellent.  A program which
   performs well and interfaces well to users is cuspy.	 See RUDE.

DAEMON (day'mun, dee'mun) [archaic form of "demon", which has slightly
   different connotations (q.v.)] n. A program which is not invoked
   explicitly, but which lays dormant waiting for some condition(s) to
   occur.  The idea is that the perpetrator of the condition need not
   be aware that a daemon is lurking (though often a program will
   commit an action only because it knows that it will implicitly
   invoke a daemon).  For example, writing a file on the lpt spooler's
   directory will invoke the spooling daemon, which prints the file.
   The advantage is that programs which want (in this example) files
   printed need not compete for access to the lpt.  They simply enter
   their implicit requests and let the daemon decide what to do with
   them.  Daemons are usually spawned automatically by the system, and
   may either live forever or be regenerated at intervals.  Usage:
   DAEMON and DEMON (q.v.) are often used interchangeably, but seem to
   have distinct connotations.	DAEMON was introduced to computing by
   CTSS people (who pronounced it dee'mon) and used it to refer to
   what is now called a DRAGON or PHANTOM (q.v.).  The meaning and
   pronunciation have drifted, and we think this glossary reflects
   current usage.

DAY MODE  See PHASE (of people).

DEADLOCK n. A situation wherein two or more processes are unable to
   proceed because each is waiting for another to do something.	 A
   common example is a program communicating to a PTY or STY, which
   may find itself waiting for output from the PTY/STY before sending
   anything more to it, while the PTY/STY is similarly waiting for
   more input from the controlling program before outputting anything.
   (This particular flavor of deadlock is called "starvation".
   Another common flavor is "constipation", where each process is
   trying to send stuff to the other, but all buffers are full because
   nobody is reading anything.)	 See DEADLY EMBRACE.

DEADLY EMBRACE n. Same as DEADLOCK (q.v.), though usually used only
   when exactly two processes are involved.  DEADLY EMBRACE is the
   more popular term in Europe; DEADLOCK in the United States.

DEMENTED adj. Yet another term of disgust used to describe a program.
   The connotation in this case is that the program works as designed,
   but the design is bad.  For example, a program that generates large
   numbers of meaningless error messages implying it is on the point
   of imminent collapse.

DEMON (dee'mun) n. A portion of a program which is not invoked
   explicitly, but which lays dormant waiting for some condition(s) to
   occur.  See DAEMON.	The distinction is that demons are usually
   processes within a program, while daemons are usually programs
   running on an operating system.  Demons are particularly common in
   AI programs.	 For example, a knowledge manipulation program might
   implement inference rules as demons.	 Whenever a new piece of
   knowledge was added, various demons would activate (which demons
   depends on the particular piece of data) and would create
   additional pieces of knowledge by applying their respective
   inference rules to the original piece.  These new pieces could in
   turn activate more demons as the inferences filtered down through
   chains of logic.  Meanwhile the main program could continue with
   whatever its primary task was.

DIABLO (dee-ah'blow) [from the Diablo printer] 1. n. Any letter-
   quality printing device.  2. v. To produce letter-quality output
   from such a device.

DIDDLE v. To work with in a not particularly serious manner.  "I
   diddled with a copy of ADVENT so it didn't double-space all the
   time."  "Let's diddle this piece of code and see if the problem
   goes away."	See TWEAK and TWIDDLE.

DIKE [from "diagonal cutters"] v. To remove a module or disable it.
   "When in doubt, dike it out."

DMP (dump)  See BIN.

DO PROTOCOL [from network protocol programming] v. To perform an
   interaction with somebody or something that follows a clearly
   defined procedure.  For example, "Let's do protocol with the check"
   at a restaurant means to ask the waitress for the check, calculate
   the tip and everybody's share, generate change as necessary, and
   pay the bill.

DOWN 1. adj. Not working.  "The up escalator is down."	2. TAKE DOWN,
   BRING DOWN: v. To deactivate, usually for repair work.  See UP.

DPB (duh-pib') [from the PDP-10 instruction set] v. To plop something
   down in the middle.

DRAGON n. (MIT) A program similar to a "daemon" (q.v.), except that it
   is not invoked at all, but is instead used by the system to perform
   various secondary tasks.  A typical example would be an accounting
   program, which keeps track of who is logged in, accumulates load-
   average statistics, etc.  At MIT, all free TV's display a list of
   people logged in, where they are, what they're running, etc. along
   with some random picture (such as a unicorn, Snoopy, or the
   Enterprise) which is generated by the "NAME DRAGON".	 See PHANTOM.

DWIM [Do What I Mean] 1. adj. Able to guess, sometimes even correctly,
   what result was intended when provided with bogus input.  Often
   suggested in jest as a desired feature for a complex program.  A
   related term, more often seen as a verb, is DTRT (Do The Right
   Thing).  2. n. The INTERLISP function that attempts to accomplish
   this feat by correcting many of the more common errors.  See HAIRY.

ENGLISH n. The source code for a program, which may be in any
   language, as opposed to BINARY.  Usage: slightly obsolete, used
   mostly by old-time hackers, though recognizable in context.	At
   MIT, directory SYSENG is where the "English" for system programs is
   kept, and SYSBIN, the binaries.  SAIL has many such directories,
   but the canonical one is [CSP,SYS].

EPSILON [from standard mathematical notation for a small quantity] 1.
   n. A small quantity of anything.  "The cost is epsilon."  2. adj.
   Very small, negligible; less than marginal (q.v.).  "We can get
   this feature for epsilon cost."  3. WITHIN EPSILON OF: Close enough
   to be indistinguishable for all practical purposes.

EXCH (ex'chuh, ekstch) [from the PDP-10 instruction set] v. To
   exchange two things, each for the other.

EXCL (eks'cul) n. Abbreviation for "exclamation point".	 See BANG,
   SHRIEK, WOW.

EXE (ex'ee)  See BIN.

FAULTY adj. Same denotation as "bagbiting", "bletcherous", "losing",
   q.v., but the connotation is much milder.

FEATURE n. 1. A surprising property of a program.  Occasionally docu-
   mented.  To call a property a feature sometimes means the author of
   the program did not consider the particular case, and the program
   makes an unexpected, although not strictly speaking an incorrect
   response.  See BUG.	"That's not a bug, that's a feature!"  A bug
   can be changed to a feature by documenting it.  2. A well-known and
   beloved property; a facility.  Sometimes features are planned, but
   are called crocks by others.	 An approximately correct spectrum:

   (These terms are all used to describe programs or portions thereof,
   except for the first two, which are included for completeness.)
	CRASH  STOPPAGE	 BUG  SCREW  LOSS  MISFEATURE
		CROCK  KLUGE  HACK  WIN	 FEATURE  PERFECTION
   (The last is never actually attained.)

FEEP 1. n. The soft bell of a display terminal (except for a VT-52!);
   a beep.  2. v. To cause the display to make a feep sound.  TTY's do
   not have feeps.  Alternate forms: BEEP, BLEEP, or just about
   anything suitably onomatopoeic.  The term BREEDLE is sometimes
   heard at SAIL, where the terminal bleepers are not particularly
   "soft" (they sound more like the musical equivalent of sticking out
   one's tongue).  The "feeper" on a VT-52 has been compared to the
   sound of a '52 Chevy stripping its gears.

FENCEPOST ERROR n. The discrete equivalent of a boundary condition.
   Often exhibited in programs by iterative loops.  From the following
   problem: "If you build a fence 100 feet long with posts ten feet
   apart, how many posts do you need?"	(Either 9 or 11 is a better
   answer than the obvious 10.)

FINE (WPI) adj. Good, but not good enough to be CUSPY.	[The word FINE
   is used elsewhere, of course, but without the implicit comparison
   to the higher level implied by CUSPY.]

FLAG DAY [from a bit of Multics history involving a change in the
   ASCII character set originally scheduled for June 14, 1966]
   n. A software change which is neither forward nor backward
   compatible, and which is costly to make and costly to revert.
   "Can we install that without causing a flag day for all users?"

FLAKEY adj. Subject to frequent lossages.  See LOSSAGE.

FLAME v. To speak incessantly and/or rabidly on some relatively
   uninteresting subject or with a patently ridiculous attitude.
   FLAME ON: v. To continue to flame.  See RAVE.  This punning
   reference to Marvel comics' Human Torch has been lost as
   recent usage completes the circle:  "Flame on" now usually
   means "beginning of flame".

FLAP v. To unload a DECtape (so it goes flap, flap, flap...).  Old
   hackers at MIT tell of the days when the disk was device 0 and
   microtapes were 1, 2,... and attempting to flap device 0 would
   instead start a motor banging inside a cabinet near the disk!

FLAVOR n. 1. Variety, type, kind.  "DDT commands come in two flavors."
   See VANILLA.	 2. The attribute of causing something to be
   FLAVORFUL.  "This convention yields additional flavor by allowing
   one to..."  3. On the LispMachine, an object-oriented programming
   system ("flavors"); each class of object is a flavor.

FLAVORFUL adj. Aesthetically pleasing.	See RANDOM and LOSING for
   antonyms.  See also the entry for TASTE.

FLUSH v. 1. To delete something, usually superfluous.  "All that
   nonsense has been flushed."	Standard ITS terminology for aborting
   an output operation.	 2. To leave at the end of a day's work (as
   opposed to leaving for a meal).  "I'm going to flush now."  "Time
   to flush."  3. To exclude someone from an activity.

FOO 1. [from Yiddish "feh" or the Anglo-Saxon "fooey!"] interj. Term
   of disgust.	2. [from FUBAR (Fucked Up Beyond All Recognition),
   from WWII, often seen as FOOBAR] Name used for temporary programs,
   or samples of three-letter names.  Other similar words are BAR, BAZ
   (Stanford corruption of BAR), and rarely RAG.  These have been used
   in Pogo as well.  3. Used very generally as a sample name for
   absolutely anything.	 The old `Smokey Stover' comic strips often
   included the word FOO, in particular on license plates of cars.
   MOBY FOO: See MOBY.

FRIED adj. 1. Non-working due to hardware failure; burnt out.  2. Of
   people, exhausted.  Said particularly of those who continue to work
   in such a state.  Often used as an explanation or excuse.  "Yeah, I
   know that fix destroyed the file system, but I was fried when I put
   it in."

FROB 1. n. (MIT) The official Tech Model Railroad Club definition is
   "FROB = protruding arm or trunnion", and by metaphoric extension
   any somewhat small thing.  See FROBNITZ.  2. v. Abbreviated form of
   FROBNICATE.

FROBNICATE v. To manipulate or adjust, to tweak.  Derived from
   FROBNITZ (q.v.).  Usually abbreviated to FROB.  Thus one has the
   saying "to frob a frob".  See TWEAK and TWIDDLE.  Usage: FROB,
   TWIDDLE, and TWEAK sometimes connote points along a continuum.
   FROB connotes aimless manipulation; TWIDDLE connotes gross
   manipulation, often a coarse search for a proper setting; TWEAK
   connotes fine-tuning.  If someone is turning a knob on an
   oscilloscope, then if he's carefully adjusting it he is probably
   tweaking it; if he is just turning it but looking at the screen he
   is probably twiddling it; but if he's just doing it because turning
   a knob is fun, he's frobbing it.

FROBNITZ, pl. FROBNITZEM (frob'nitsm) n. An unspecified physical
   object, a widget.  Also refers to electronic black boxes.  This
   rare form is usually abbreviated to FROTZ, or more commonly to
   FROB.  Also used are FROBNULE, FROBULE, and FROBNODULE.  Starting
   perhaps in 1979, FROBBOZ (fruh-bahz'), pl. FROBBOTZIM, has also
   become very popular, largely due to its exposure via the Adventure
   spin-off called Zork (Dungeon).  These can also be applied to
   non-physical objects, such as data structures.

FROG (variant: PHROG) 1. interj. Term of disgust (we seem to have a
   lot of them).  2. Used as a name for just about anything.  See FOO.
   3. n. Of things, a crock.  Of people, somewhere inbetween a turkey
   and a toad.	4. Jake Brown (FRG@SAIL).  5. FROGGY: adj. Similar to
   BAGBITING (q.v.), but milder.  "This froggy program is taking
   forever to run!"

FROTZ 1. n. See FROBNITZ.  2. MUMBLE FROTZ: An interjection of very
   mild disgust.

FRY v. 1. To fail.  Said especially of smoke-producing hardware
   failures.  2. More generally, to become non-working.	 Usage: never
   said of software, only of hardware and humans.  See FRIED.

FTP (spelled out, NOT pronounced "fittip") 1. n. The File Transfer
   Protocol for transmitting files between systems on the ARPAnet.  2.
   v. To transfer a file using the File Transfer Program.  "Lemme get
   this copy of Wuthering Heights FTP'd from SAIL."

FUDGE 1. v. To perform in an incomplete but marginally acceptable way,
   particularly with respect to the writing of a program.  "I didn't
   feel like going through that pain and suffering, so I fudged it."
   2. n. The resulting code.

FUDGE FACTOR n. A value or parameter that is varied in an ad hoc way
   to produce the desired result.  The terms "tolerance" and "slop"
   are also used, though these usually indicate a one-sided leeway,
   such as a buffer which is made larger than necessary because one
   isn't sure exactly how large it needs to be, and it is better to
   waste a little space than to lose completely for not having enough.
   A fudge factor, on the other hand, can often be tweaked in more
   than one direction.	An example might be the coefficients of an
   equation, where the coefficients are varied in an attempt to make
   the equation fit certain criteria.

GABRIEL [for Dick Gabriel, SAIL volleyball fanatic] n. An unnecessary
   (in the opinion of the opponent) stalling tactic, e.g., tying one's
   shoelaces or hair repeatedly, asking the time, etc.	Also used to
   refer to the perpetrator of such tactics.  Also, "pulling a
   Gabriel", "Gabriel mode".

GARBAGE COLLECT v., GARBAGE COLLECTION n. See GC.

GARPLY n. (Stanford) Another meta-word popular among SAIL hackers.

GAS [as in "gas chamber"] interj. 1. A term of disgust and hatred,
   implying that gas should be dispensed in generous quantities,
   thereby exterminating the source of irritation.  "Some loser just
   reloaded the system for no reason!  Gas!"  2. A term suggesting
   that someone or something ought to be flushed out of mercy.	"The
   system's wedging every few minutes.	Gas!"  3. v. FLUSH (q.v.).
   "You should gas that old crufty software."  4. GASEOUS adj.
   Deserving of being gassed.  Usage: primarily used by Geoff
   Goodfellow at SRI, but spreading.

GC [from LISP terminology] 1. v. To clean up and throw away useless
   things.  "I think I'll GC the top of my desk today."	 2. To
   recycle, reclaim, or put to another use.  3. To forget.  The
   implication is often that one has done so deliberately.  4. n. An
   instantiation of the GC process.

GEDANKEN [from Einstein's term "gedanken-experimenten", such as the
   standard proof that E=mc^2] adj. An AI project which is written up
   in grand detail without ever being implemented to any great extent.
   Usually perpetrated by people who aren't very good hackers or find
   programming distasteful or are just in a hurry.  A gedanken thesis
   is usually marked by an obvious lack of intuition about what is
   programmable and what is not and about what does and does not
   constitute a clear specification of a program-related concept such
   as an algorithm.

GLASS TTY n. A terminal which has a display screen but which, because
   of hardware or software limitations, behaves like a teletype or
   other printing terminal.  An example is the ADM-3 (without cursor
   control).  A glass tty can't do neat display hacks, and you can't
   save the output either.

GLITCH [from the Yiddish "glitshen", to slide] 1. n. A sudden
   interruption in electric service, sanity, or program function.
   Sometimes recoverable.  2. v. To commit a glitch.  See GRITCH.
   3. v. (Stanford) To scroll a display screen.

GLORK 1. interj. Term of mild surprise, usually tinged with outrage,
   as when one attempts to save the results of two hours of editing
   and finds that the system has just crashed.	2. Used as a name for
   just about anything.	 See FOO.  3. v. Similar to GLITCH (q.v.), but
   usually used reflexively.  "My program just glorked itself."

GOBBLE v. To consume or to obtain.  GOBBLE UP tends to imply
   "consume", while GOBBLE DOWN tends to imply "obtain".  "The output
   spy gobbles characters out of a TTY output buffer."	"I guess I'll
   gobble down a copy of the documentation tomorrow."  See SNARF.

GORP (CMU) [perhaps from the generic term for dried hiker's food,
   stemming from the acronym "Good Old Raisins and Peanuts"] Another
   metasyntactic variable, like FOO and BAR.

GRIND v. 1. (primarily MIT) To format code, especially LISP code, by
   indenting lines so that it looks pretty.  Hence, PRETTY PRINT, the
   generic term for such operations.  2. To run seemingly
   interminably, performing some tedious and inherently useless task.
   Similar to CRUNCH.

GRITCH 1. n. A complaint (often caused by a GLITCH (q.v.)).  2. v. To
   complain.  Often verb-doubled: "Gritch gritch".  3. Glitch.

GROK [from the novel "Stranger in a Strange Land", by Robert Heinlein,
   where it is a Martian word meaning roughly "to be one with"] v. To
   understand, usually in a global sense.

GRONK [popularized by the cartoon strip "B.C." by Johnny Hart, but the
   word apparently predates that] v. 1. To clear the state of a wedged
   device and restart it.  More severe than "to frob" (q.v.).  2. To
   break.  "The teletype scanner was gronked, so we took the system
   down."  3. GRONKED: adj. Of people, the condition of feeling very
   tired or sick.  4. GRONK OUT: v. To cease functioning.  Of people,
   to go home and go to sleep.	"I guess I'll gronk out now; see you
   all tomorrow."

GROVEL v. To work interminably and without apparent progress.  Often
   used with "over".  "The compiler grovelled over my code."  Compare
   GRIND and CRUNCH.  Emphatic form: GROVEL OBSCENELY.

GRUNGY adj. Incredibly dirty or grubby.	 Anything which has been
   washed within the last year is not really grungy.  Also used
   metaphorically; hence some programs (especially crocks) can be
   described as grungy.

GUBBISH [a portmanteau of "garbage" and "rubbish"?] n. Garbage; crap;
   nonsense.  "What is all this gubbish?"

GUN [from the GUN command on ITS] v. To forcibly terminate a program
   or job (computer, not career).  "Some idiot left a background
   process running soaking up half the cycles, so I gunned it."

HACK n. 1. Originally a quick job that produces what is needed, but
   not well.  2. The result of that job.  3. NEAT HACK: A clever
   technique.  Also, a brilliant practical joke, where neatness is
   correlated with cleverness, harmlessness, and surprise value.
   Example: the Caltech Rose Bowl card display switch circa 1961.
   4. REAL HACK: A crock (occasionally affectionate).
   v. 5. With "together", to throw something together so it will work.
   6. To bear emotionally or physically.  "I can't hack this heat!" 7.
   To work on something (typically a program).	In specific sense:
   "What are you doing?"  "I'm hacking TECO."  In general sense: "What
   do you do around here?"  "I hack TECO."  (The former is
   time-immediate, the latter time-extended.)  More generally, "I hack
   x" is roughly equivalent to "x is my bag".  "I hack solid-state
   physics."  8. To pull a prank on.  See definition 3 and HACKER (def
   #6).	 9. v.i. To waste time (as opposed to TOOL).  "Watcha up to?"
   "Oh, just hacking."	10. HACK UP (ON): To hack, but generally
   implies that the result is meanings 1-2.  11. HACK VALUE: Term used
   as the reason or motivation for expending effort toward a seemingly
   useless goal, the point being that the accomplished goal is a hack.
   For example, MacLISP has code to read and print roman numerals,
   which was installed purely for hack value.
   HAPPY HACKING: A farewell.  HOW'S HACKING?: A friendly greeting
   among hackers.  HACK HACK: A somewhat pointless but friendly
   comment, often used as a temporary farewell.
   [The word HACK doesn't really have 69 different meanings.  In fact,
   HACK has only one meaning, an extremely subtle and profound one
   which defies articulation.  Which connotation a given HACK-token
   has depends in similarly profound ways on the context.  Similar
   comments apply to a couple other hacker jargon items, most notably
   RANDOM. - Agre]

HACKER [originally, someone who makes furniture with an axe] n. 1. A
   person who enjoys learning the details of programming systems and
   how to stretch their capabilities, as opposed to most users who
   prefer to learn only the minimum necessary.	2. One who programs
   enthusiastically, or who enjoys programming rather than just
   theorizing about programming.  3. A person capable of appreciating
   hack value (q.v.).  4. A person who is good at programming quickly.
   Not everything a hacker produces is a hack.	5. An expert at a
   particular program, or one who frequently does work using it or on
   it; example: "A SAIL hacker".  (Definitions 1 to 5 are correlated,
   and people who fit them congregate.)	 6. A malicious or inquisitive
   meddler who tries to discover information by poking around.	Hence
   "password hacker", "network hacker".

HACKISH adj. Being or involving a hack.	 HACKISHNESS n.

HAIR n. The complications which make something hairy.  "Decoding TECO
   commands requires a certain amount of hair."	 Often seen in the
   phrase INFINITE HAIR, which connotes extreme complexity.

HAIRY adj. 1. Overly complicated.  "DWIM is incredibly hairy."	2.
   Incomprehensible.  "DWIM is incredibly hairy."  3.  Of people,
   high-powered, authoritative, rare, expert, and/or incomprehensible.
   Hard to explain except in context: "He knows this hairy lawyer who
   says there's nothing to worry about."

HAKMEM n. MIT AI Memo 239 (February 1972).  A collection of neat
   mathematical and programming hacks contributed by many people
   at MIT and elsewhere.

HANDWAVE 1. v. To gloss over a complex point; to distract a listener;
   to support a (possibly actually valid) point with blatantly faulty
   logic.  2. n. The act of handwaving.	 "Boy, what a handwave!"  The
   use of this word is often accompanied by gestures: both hands up,
   palms forward, swinging the hands in a vertical plane pivoting at
   the elbows and/or shoulders (depending on the magnitude of the
   handwave); alternatively, holding the forearms still while rotating
   the hands at the wrist to make them flutter.	 In context, the
   gestures alone can suffice as a remark.

HARDWARILY adv. In a way pertaining to hardware.  "The system is
   hardwarily unreliable."  The adjective "hardwary" is NOT used.  See
   SOFTWARILY.

HELLO WALL  See WALL.

HIRSUTE	 Occasionally used humorously as a synonym for HAIRY.

HOOK n. An extraneous piece of software or hardware included in order
   to simplify later additions or debug options.  For instance, a
   program might execute a location that is normally a JFCL, but by
   changing the JFCL to a PUSHJ one can insert a debugging routine at
   that point.

HUMONGOUS, HUMUNGOUS  See HUNGUS.

HUNGUS (hung'ghis) [perhaps related to current slang "humongous";
   which one came first (if either) is unclear] adj. Large, unwieldy,
   usually unmanageable.  "TCP is a hungus piece of code."  "This is a
   hungus set of modifications."

IMPCOM	See TELNET.

INFINITE adj. Consisting of a large number of objects; extreme.	 Used
   very loosely as in: "This program produces infinite garbage."

IRP (erp) [from the MIDAS pseudo-op which generates a block of code
   repeatedly, substituting in various places the car and/or cdr of
   the list(s) supplied at the IRP] v. To perform a series of tasks
   repeatedly with a minor substitution each time through.  "I guess
   I'll IRP over these homework papers so I can give them some random
   grade for this semester."

JFCL (djif'kl or dja-fik'l) [based on the PDP-10 instruction that acts
   as a fast no-op] v. To cancel or annul something.  "Why don't you
   jfcl that out?"  [The licence plate on Geoff Goodfellow's BMW is
   JFCL.]

JIFFY n. 1. Interval of CPU time, commonly 1/60 second or 1
   millisecond.	 2. Indeterminate time from a few seconds to forever.
   "I'll do it in a jiffy" means certainly not now and possibly never.

JOCK n. Programmer who is characterized by large and somewhat brute
   force programs.  The term is particularly well-suited for systems
   programmers.

J. RANDOM  See RANDOM.

JRST (jerst) [based on the PDP-10 jump instruction] v. To suddenly
   change subjects.  Usage: rather rare.  "Jack be nimble, Jack be
   quick; Jack jrst over the candle stick."

JSYS (jay'sis), pl. JSI (jay'sigh) [Jump to SYStem] See UUO.

KLUGE (kloodj) alt. KLUDGE [from the German "kluge", clever] n. 1. A
   Rube Goldberg device in hardware or software.  2. A clever
   programming trick intended to solve a particular nasty case in an
   efficient, if not clear, manner.  Often used to repair bugs.	 Often
   verges on being a crock.  3. Something that works for the wrong
   reason.  4. v. To insert a kluge into a program.  "I've kluged this
   routine to get around that weird bug, but there's probably a better
   way."  Also KLUGE UP.  5. KLUGE AROUND: To avoid by inserting a
   kluge.  6. (WPI) A feature which is implemented in a RUDE manner.

LDB (lid'dib) [from the PDP-10 instruction set] v. To extract from the
   middle.

LIFE n. A cellular-automata game invented by John Horton Conway, and
   first introduced publicly by Martin Gardner (Scientific American,
   October 1970).

LINE FEED (standard ASCII terminology) 1. v. To feed the paper through
   a terminal by one line (in order to print on the next line).	 2. n.
   The "character" which causes the terminal to perform this action.

LINE STARVE (MIT) Inverse of LINE FEED.

LOGICAL [from the technical term "logical device", wherein a physical
   device is referred to by an arbitrary name] adj. Understood to have
   a meaning not necessarily corresponding to reality.	E.g., if a
   person who has long held a certain post (e.g., Les Earnest at SAIL)
   left and was replaced, the replacement would for a while be known
   as the "logical Les Earnest".  The word VIRTUAL is also used.  At
   SAIL, "logical" compass directions denote a coordinate system in
   which "logical north" is toward San Francisco, "logical west" is
   toward the ocean, etc., even though logical north varies between
   physical (true) north near SF and physical west near San Jose.
   (The best rule of thumb here is that El Camino Real by definition
   always runs logical north-and-south.)

LOSE [from MIT jargon] v. 1. To fail.  A program loses when it
   encounters an exceptional condition.	 2. To be exceptionally
   unaesthetic.	 3. Of people, to be obnoxious or unusually stupid (as
   opposed to ignorant).  4. DESERVE TO LOSE: v. Said of someone who
   willfully does the wrong thing; humorously, if one uses a feature
   known to be marginal.  What is meant is that one deserves the
   consequences of one's losing actions.  "Boy, anyone who tries to
   use MULTICS deserves to lose!"
   LOSE LOSE - a reply or comment on a situation.

LOSER n. An unexpectedly bad situation, program, programmer, or
   person.  Especially "real loser".

LOSS n. Something which loses.	WHAT A (MOBY) LOSS!: interjection.

LOSSAGE n. The result of a bug or malfunction.

LPT (lip'it) n. Line printer, of course.

LUSER  See USER.

MACROTAPE n. An industry standard reel of tape, as opposed to a
   MICROTAPE.

MAGIC adj. 1. As yet unexplained, or too complicated to explain.
   (Arthur C. Clarke once said that magic was as-yet-not-understood
   science.)  "TTY echoing is controlled by a large number of magic
   bits."  "This routine magically computes the parity of an eight-bit
   byte in three instructions."	 2. (Stanford) A feature not generally
   publicized which allows something otherwise impossible, or a
   feature formerly in that category but now unveiled.	Example: The
   keyboard commands which override the screen-hiding features.

MARGINAL adj. 1. Extremely small.  "A marginal increase in core can
   decrease GC time drastically."  See EPSILON.	 2. Of extremely small
   merit.  "This proposed new feature seems rather marginal to me."
   3. Of extremely small probability of winning.  "The power supply
   was rather marginal anyway; no wonder it crapped out."  4.
   MARGINALLY: adv. Slightly.  "The ravs here are only marginally
   better than at Small Eating Place."

MICROTAPE n. Occasionally used to mean a DECtape, as opposed to a
   MACROTAPE.  This was the official DEC term for the stuff until
   someone consed up the word "DECtape".

MISFEATURE n. A feature which eventually screws someone, possibly
   because it is not adequate for a new situation which has evolved.
   It is not the same as a bug because fixing it involves a gross
   philosophical change to the structure of the system involved.
   Often a former feature becomes a misfeature because a tradeoff was
   made whose parameters subsequently changed (possibly only in the
   judgment of the implementors).  "Well, yeah, it's kind of a
   misfeature that file names are limited to six characters, but we're
   stuck with it for now."

MOBY [seems to have been in use among model railroad fans years ago.
   Entered the world of AI with the Fabritek 256K moby memory of
   MIT-AI.  Derived from Melville's "Moby Dick" (some say from "Moby
   Pickle").] 1. adj. Large, immense, or complex.  "A moby frob."  2.
   n. The maximum address space of a machine, hence 3. n. 256K words,
   the size of a PDP-10 moby.  (The maximum address space means the
   maximum normally addressable space, as opposed to the amount of
   physical memory a machine can have.	Thus the MIT PDP-10s each have
   two mobies, usually referred to as the "low moby" (0-777777) and
   "high moby" (1000000-1777777), or as "moby 0" and "moby 1".	MIT-AI
   has four mobies of address space: moby 2 is the PDP-6 memory, and
   moby 3 the PDP-11 interface.)  In this sense "moby" is often used
   as a generic unit of either address space (18. bits' worth) or of
   memory (about a megabyte, or 9/8 megabyte (if one accounts for
   difference between 32.- and 36.-bit words), or 5/4 megacharacters).
   4. A title of address (never of third-person reference), usually
   used to show admiration, respect, and/or friendliness to a
   competent hacker.  "So, moby Knight, how's the CONS machine doing?"
   5. adj. In backgammon, doubles on the dice, as in "moby sixes",
   "moby ones", etc.
   MOBY FOO, MOBY WIN, MOBY LOSS: standard emphatic forms.
   FOBY MOO: a spoonerism due to Greenblatt.

MODE n. A general state, usually used with an adjective describing the
   state.  "No time to hack; I'm in thesis mode."  Usage: in its
   jargon sense, MODE is most often said of people, though it is
   sometimes applied to programs and inanimate objects.	 "If you're on
   a TTY, E will switch to non-display mode."  In particular, see DAY
   MODE, NIGHT MODE, and YOYO MODE; also COM MODE, TALK MODE, and
   GABRIEL MODE.

MODULO prep. Except for.  From mathematical terminology: one can
   consider saying that 4=22 "except for the 9's" (4=22 mod 9).
   "Well, LISP seems to work okay now, modulo that GC bug."

MOON n. 1. A celestial object whose phase is very important to
   hackers.  See PHASE OF THE MOON.  2. Dave Moon (MOON@MC).

MUMBLAGE n. The topic of one's mumbling (see MUMBLE).  "All that
   mumblage" is used like "all that stuff" when it is not quite clear
   what it is or how it works, or like "all that crap" when "mumble"
   is being used as an implicit replacement for obscenities.

MUMBLE interj. 1. Said when the correct response is either too
   complicated to enunciate or the speaker has not thought it out.
   Often prefaces a longer answer, or indicates a general reluctance
   to get into a big long discussion.  "Well, mumble."	2. Sometimes
   used as an expression of disagreement.  "I think we should buy it."
   "Mumble!"  Common variant: MUMBLE FROTZ.  3. Yet another
   metasyntactic variable, like FOO.

MUNCH (often confused with "mung", q.v.) v. To transform information
   in a serial fashion, often requiring large amounts of computation.
   To trace down a data structure.  Related to CRUNCH (q.v.), but
   connotes less pain.

MUNCHING SQUARES n. A display hack dating back to the PDP-1, which
   employs a trivial computation (involving XOR'ing of x-y display
   coordinates - see HAKMEM items 146-148) to produce an impressive
   display of moving, growing, and shrinking squares.  The hack
   usually has a parameter (usually taken from toggle switches) which
   when well-chosen can produce amazing effects.  Some of these,
   discovered recently on the LISP machine, have been christened
   MUNCHING TRIANGLES, MUNCHING W'S, and MUNCHING MAZES.

MUNG (variant: MUNGE) [recursive acronym for Mung Until No Good] v. 1.
   To make changes to a file, often large-scale, usually irrevocable.
   Occasionally accidental.  See BLT.  2. To destroy, usually
   accidentally, occasionally maliciously.  The system only mungs
   things maliciously.

N adj. 1. Some large and indeterminate number of objects; "There were
   N bugs in that crock!"; also used in its original sense of a
   variable name.  2. An arbitrarily large (and perhaps infinite)
   number.  3. A variable whose value is specified by the current
   context.  "We'd like to order N wonton soups and a family dinner
   for N-1."  4. NTH: adj. The ordinal counterpart of N. "Now for the
   Nth and last time..."  In the specific context "Nth-year grad
   student", N is generally assumed to be at least 4, and is usually 5
   or more.  See also 69.

NIGHT MODE  See PHASE (of people).

NIL [from LISP terminology for "false"] No.  Usage: used in reply to a
   question, particularly one asked using the "-P" convention.	See T.

OBSCURE adj. Used in an exaggeration of its normal meaning, to imply a
   total lack of comprehensibility.  "The reason for that last crash
   is obscure."	 "FIND's command syntax is obscure."  MODERATELY
   OBSCURE implies that it could be figured out but probably isn't
   worth the trouble.

OPEN n. Abbreviation for "open (or left) parenthesis", used when
   necessary to eliminate oral ambiguity.  To read aloud the LISP form
   (DEFUN FOO (X) (PLUS X 1)) one might say: "Open def-fun foo, open
   eks close, open, plus ekx one, close close."	 See CLOSE.

PARSE [from linguistic terminology] v. 1. To determine the syntactic
   structure of a sentence or other utterance (close to the standard
   English meaning).  Example: "That was the one I saw you."  "I can't
   parse that."	 2. More generally, to understand or comprehend.
   "It's very simple; you just kretch the glims and then aos the
   zotz."  "I can't parse that."  3. Of fish, to have to remove the
   bones yourself (usually at a Chinese restaurant).  "I object to
   parsing fish" means "I don't want to get a whole fish, but a sliced
   one is okay."  A "parsed fish" has been deboned.  There is some
   controversy over whether "unparsed" should mean "bony", or also
   mean "deboned".

PATCH 1. n. A temporary addition to a piece of code, usually as a
   quick-and-dirty remedy to an existing bug or misfeature.  A patch
   may or may not work, and may or may not eventually be incorporated
   permanently into the program.  2. v. To insert a patch into a piece
   of code.

PDL (piddle or puddle) [acronym for Push Down List] n. 1. A LIFO queue
   (stack); more loosely, any priority queue; even more loosely, any
   queue.  A person's pdl is the set of things he has to do in the
   future.  One speaks of the next project to be attacked as having
   risen to the top of the pdl.	 "I'm afraid I've got real work to do,
   so this'll have to be pushed way down on my pdl."  See PUSH and
   POP.	 2. Dave Lebling (PDL@DM).

PESSIMAL [Latin-based antonym for "optimal"] adj. Maximally bad.
   "This is a pessimal situation."

PESSIMIZING COMPILER n. A compiler that produces object code that is
   worse than the straightforward or obvious translation.

PHANTOM n. (Stanford) The SAIL equivalent of a DRAGON (q.v.).  Typical
   phantoms include the accounting program, the news-wire monitor, and
   the lpt and xgp spoolers.

PHASE (of people) 1. n. The phase of one's waking-sleeping schedule
   with respect to the standard 24-hour cycle.	This is a useful
   concept among people who often work at night according to no fixed
   schedule.  It is not uncommon to change one's phase by as much as
   six hours/day on a regular basis.  "What's your phase?"  "I've been
   getting in about 8 PM lately, but I'm going to work around to the
   day schedule by Friday."  A person who is roughly 12 hours out of
   phase is sometimes said to be in "night mode".  (The term "day
   mode" is also used, but less frequently.)  2. CHANGE PHASE THE HARD
   WAY: To stay awake for a very long time in order to get into a
   different phase.  3. CHANGE PHASE THE EASY WAY: To stay asleep etc.

PHASE OF THE MOON n. Used humorously as a random parameter on which
   something is said to depend.	 Sometimes implies unreliability of
   whatever is dependent, or that reliability seems to be dependent on
   conditions nobody has been able to determine.  "This feature
   depends on having the channel open in mumble mode, having the foo
   switch set, and on the phase of the moon."

PLUGH [from the Adventure game] v. See XYZZY.

POM n. Phase of the moon (q.v.).  Usage: usually used in the phrase
   "POM dependent" which means flakey (q.v.).

POP [based on the stack operation that removes the top of a stack, and
   the fact that procedure return addresses are saved on the stack]
   dialect: POPJ (pop-jay), based on the PDP-10 procedure return
   instruction.	 v. To return from a digression.  By verb doubling,
   "Popj, popj" means roughly, "Now let's see, where were we?"

PPN (pip'in) [DEC terminology, short for Project-Programmer Number] n.
   1. A combination `project' (directory name) and programmer name,
   used to identify a specific directory belonging to that user.  For
   instance, "FOO,BAR" would be the FOO directory for user BAR.	 Since
   the name is restricted to three letters, the programmer name is
   usually the person's initials, though sometimes it is a nickname or
   other special sequence.  (Standard DEC setup is to have two octal
   numbers instead of characters; hence the original acronym.)	2.
   Often used loosely to refer to the programmer name alone.  "I want
   to send you some mail; what's your ppn?"  Usage: not used at MIT,
   since ITS does not use ppn's.  The equivalent terms would be UNAME
   and SNAME, depending on context, but these are not used except in
   their technical senses.

PROTOCOL  See DO PROTOCOL.

PSEUDOPRIME n. A backgammon prime (six consecutive occupied points)
   with one point missing.

PTY (pity) n. Pseudo TTY, a simulated TTY used to run a job under the
   supervision of another job.
   PTYJOB (pity-job) n. The job being run on the PTY.  Also a common
   general-purpose program for creating and using PTYs.
   This is DEC and SAIL terminology; the MIT equivalent is STY.

PUNT [from the punch line of an old joke: "Drop back 15 yards and
   punt"] v. To give up, typically without any intention of retrying.

PUSH [based on the stack operation that puts the current information
   on a stack, and the fact that procedure call addresses are saved on
   the stack] dialect: PUSHJ (push-jay), based on the PDP-10 procedure
   call instruction.  v. To enter upon a digression, to save the
   current discussion for later.

QUES (kwess) 1. n. The question mark character ("?").  2. interj.
   What?  Also QUES QUES?  See WALL.

QUUX [invented by Steele.  Mythically, from the Latin semi-deponent
   verb QUUXO, QUUXARE, QUUXANDUM IRI; noun form variously QUUX
   (plural QUUCES, Anglicized to QUUXES) and QUUXU (genitive plural is
   QUUXUUM, four U's in seven letters).] 1. Originally, a meta-word
   like FOO and FOOBAR.	 Invented by Guy Steele for precisely this
   purpose when he was young and naive and not yet interacting with
   the real computing community.  Many people invent such words; this
   one seems simply to have been lucky enough to have spread a little.
   2. interj. See FOO; however, denotes very little disgust, and is
   uttered mostly for the sake of the sound of it.  3. n. Refers to
   one of four people who went to Boston Latin School and eventually
   to MIT:
	THE GREAT QUUX:	 Guy L. Steele Jr.
	THE LESSER QUUX:  David J. Littleboy
	THE MEDIOCRE QUUX:  Alan P. Swide
	THE MICRO QUUX:	 Sam Lewis
   (This taxonomy is said to be similarly applied to three Frankston
   brothers at MIT.)  QUUX, without qualification, usually refers to
   The Great Quux, who is somewhat infamous for light verse and for
   the "Crunchly" cartoons.  4. QUUXY: adj. Of or pertaining to a
   QUUX.

RANDOM adj. 1. Unpredictable (closest to mathematical definition);
   weird.  "The system's been behaving pretty randomly."  2. Assorted;
   undistinguished.  "Who was at the conference?"  "Just a bunch of
   random business types."  3.	Frivolous; unproductive; undirected
   (pejorative).  "He's just a random loser."  4. Incoherent or
   inelegant; not well organized.  "The program has a random set of
   misfeatures."  "That's a random name for that function."  "Well,
   all the names were chosen pretty randomly."	5. Gratuitously wrong,
   i.e., poorly done and for no good apparent reason.  For example, a
   program that handles file name defaulting in a particularly useless
   way, or a routine that could easily have been coded using only
   three ac's, but randomly uses seven for assorted non-overlapping
   purposes, so that no one else can invoke it without first saving
   four extra ac's.  6. In no particular order, though deterministic.
   "The I/O channels are in a pool, and when a file is opened one is
   chosen randomly."  n. 7. A random hacker; used particularly of high
   school students who soak up computer time and generally get in the
   way.	 8. (occasional MIT usage) One who lives at Random Hall.
   J. RANDOM is often prefixed to a noun to make a "name" out of it
   (by comparison to common names such as "J. Fred Muggs").  The most
   common uses are "J. Random Loser" and "J. Random Nurd" ("Should
   J. Random Loser be allowed to gun down other people?"), but it
   can be used just as an elaborate version of RANDOM in any sense.
   [See also the note at the end of the entry for HACK.]

RANDOMNESS n. An unexplainable misfeature; gratuitous inelegance.
   Also, a hack or crock which depends on a complex combination
   of coincidences (or rather, the combination upon which the
   crock depends).  "This hack can output characters 40-57 by
   putting the character in the accumulator field of an XCT and
   then extracting 6 bits -- the low two bits of the XCT opcode
   are the right thing."  "What randomness!"

RAPE v. To (metaphorically) screw someone or something, violently.
   Usage: often used in describing file-system damage.	"So-and-so was
   running a program that did absolute disk I/O and ended up raping
   the master directory."

RAVE (WPI) v. 1. To persist in discussing a specific subject.  2. To
   speak authoritatively on a subject about which one knows very
   little.  3. To complain to a person who is not in a position to
   correct the difficulty.  4. To purposely annoy another person
   verbally.  5. To evangelize.	 See FLAME.  Also used to describe
   a less negative form of blather, such as friendly bullshitting.

REAL USER n. 1. A commercial user.  One who is paying "real" money for
   his computer usage.	2. A non-hacker.  Someone using the system for
   an explicit purpose (research project, course, etc.).  See USER.

REAL WORLD, THE n. 1. In programming, those institutions at which
   programming may be used in the same sentence as FORTRAN, COBOL,
   RPG, IBM, etc.  2. To programmers, the location of non-programmers
   and activities not related to programming.  3. A universe in which
   the standard dress is shirt and tie and in which a person's working
   hours are defined as 9 to 5.	 4. The location of the status quo.
   5. Anywhere outside a university.  "Poor fellow, he's left MIT and
   gone into the real world."  Used pejoratively by those not in
   residence there.  In conversation, talking of someone who has
   entered the real world is not unlike talking about a deceased
   person.

RECURSION n. See RECURSION, TAIL RECURSION.

REL  See BIN.

RIGHT THING, THE n. That which is "obviously" the correct or
   appropriate thing to use, do, say, etc.  Use of this term often
   implies that in fact reasonable people may disagree.	 "Never let
   your conscience keep you from doing the right thing!"  "What's the
   right thing for LISP to do when it reads '(.)'?"

RUDE (WPI) adj. 1. (of a program) Badly written.  2. Functionally
   poor, e.g. a program which is very difficult to use because of
   gratuitously poor (random?) design decisions.  See CUSPY.

SACRED adj. Reserved for the exclusive use of something (a
   metaphorical extension of the standard meaning).  "Accumulator 7 is
   sacred to the UUO handler."	Often means that anyone may look at
   the sacred object, but clobbering it will screw whatever it is
   sacred to.

SAGA (WPI) n. A cuspy but bogus raving story dealing with N random
   broken people.

SAV (save)  See BIN.

SEMI 1. n. Abbreviation for "semicolon", when speaking.	 "Commands to
   GRIND are prefixed by semi-semi-star" means that the prefix is
   ";;*", not 1/4 of a star.  2. Prefix with words such as
   "immediately", as a qualifier.  "When is the system coming up?"
   "Semi-immediately."

SERVER n. A kind of DAEMON which performs a service for the requester,
   which often runs on a computer other than the one on which the
   server runs.

SHIFT LEFT (RIGHT) LOGICAL [from any of various machines' instruction
   sets] 1. v. To move oneself to the left (right).  To move out of
   the way.  2. imper. Get out of that (my) seat!  Usage: often used
   without the "logical", or as "left shift" instead of "shift left".
   Sometimes heard as LSH (lish), from the PDP-10 instruction set.

SHR (share or shir)  See BIN.

SHRIEK	See EXCL.  (Occasional CMU usage.)

69 adj. Large quantity.	 Usage: Exclusive to MIT-AI.  "Go away, I have
   69 things to do to DDT before worrying about fixing the bug in the
   phase of the moon output routine..."
   [Note: Actually, any number less than 100 but large enough to have
   no obvious magic properties will be recognized as a "large number".
   There is no denying that "69" is the local favorite.	 I don't know
   whether its origins are related to the obscene interpretation, but
   I do know that 69 decimal = 105 octal, and 69 hexadecimal = 105
   decimal, which is a nice property. - GLS]

SLOP n. 1. A one-sided fudge factor (q.v.).  Often introduced to avoid
   the possibility of a fencepost error (q.v.).	 2. (used by compiler
   freaks) The ratio of code generated by a compiler to hand-compiled
   code, minus 1; i.e., the space (or maybe time) you lose because you
   didn't do it yourself.

SLURP v. To read a large data file entirely into core before working
   on it.  "This program slurps in a 1K-by-1K matrix and does an FFT."

SMART adj. Said of a program that does the Right Thing (q.v.) in a
   wide variety of complicated circumstances.  There is a difference
   between calling a program smart and calling it intelligent; in
   particular, there do not exist any intelligent programs.

SMOKING CLOVER n. A psychedelic color munch due to Gosper.

SMOP [Simple (or Small) Matter of Programming] n. A piece of code, not
   yet written, whose anticipated length is significantly greater than
   its complexity.  Usage: used to refer to a program that could
   obviously be written, but is not worth the trouble.

SNARF v. To grab, esp. a large document or file for the purpose of
   using it either with or without the author's permission.  See BLT.
   Variant: SNARF (IT) DOWN.  (At MIT on ITS, DDT has a command called
   :SNARF which grabs a job from another (inferior) DDT.)

SOFTWARE ROT n. Hypothetical disease the existence of which has been
   deduced from the observation that unused programs or features will
   stop working after sufficient time has passed, even if "nothing has
   changed".  Also known as "bit decay".

SOFTWARILY adv. In a way pertaining to software.  "The system is
   softwarily unreliable."  The adjective "softwary" is NOT used.  See
   HARDWARILY.

SOS 1. (ess-oh-ess) n. A losing editor, SON OF STOPGAP.	 2. (sahss) v.
   Inverse of AOS, from the PDP-10 instruction set.

SPAZZ 1. v. To behave spastically or erratically; more often, to
   commit a single gross error.	 "Boy, is he spazzing!"	 2. n. One who
   spazzes.  "Boy, what a spazz!"  3. n. The result of spazzing.
   "Boy, what a spazz!"

SPLAT n. 1. Name used in many places (DEC, IBM, and others) for the
   ASCII star ("*") character.	2. (MIT) Name used by some people for
   the ASCII pound-sign ("#") character.  3. (Stanford) Name used by
   some people for the Stanford/ITS extended ASCII circle-x character.
   (This character is also called "circle-x", "blobby", and "frob",
   among other names.)	4. (Stanford) Name for the semi-mythical
   extended ASCII circle-plus character.  5. Canonical name for an
   output routine that outputs whatever the the local interpretation
   of splat is.	 Usage: nobody really agrees what character "splat"
   is, but the term is common.

SUPDUP v. To communicate with another ARPAnet host using the SUPDUP
   program, which is a SUPer-DUPer TELNET talking a special display
   protocol used mostly in talking to ITS sites.  Sometimes
   abbreviated to SD.

STATE n. Condition, situation.	"What's the state of NEWIO?"  "It's
   winning away."  "What's your state?"	 "I'm about to gronk out."  As
   a special case, "What's the state of the world?" (or, more silly,
   "State-of-world-P?") means "What's new?" or "What's going on?"

STOPPAGE n. Extreme lossage (see LOSSAGE) resulting in something
   (usually vital) becoming completely unusable.

STY (pronounced "sty", not spelled out) n. A pseudo-teletype, which is
   a two-way pipeline with a job on one end and a fake keyboard-tty on
   the other.  Also, a standard program which provides a pipeline from
   its controlling tty to a pseudo-teletype (and thence to another
   tty, thereby providing a "sub-tty").
   This is MIT terminology; the SAIL and DEC equivalent is PTY.

SUPERPROGRAMMER n. See "wizard", "hacker".  Usage: rare.  (Becoming
   more common among IBM and Yourdon types.)

SWAPPED adj. From the use of secondary storage devices to implement
   virtual memory in computer systems.	Something which is SWAPPED IN
   is available for immediate use in main memory, and otherwise is
   SWAPPED OUT.	 Often used metaphorically to refer to people's
   memories ("I read TECO ORDER every few months to keep the
   information swapped in.") or to their own availability ("I'll swap
   you in as soon as I finish looking at this other problem.").

SYSTEM n. 1. The supervisor program on the computer.  2. Any
   large-scale program.	 3. Any method or algorithm.  4. The way
   things are usually done.  Usage: a fairly ambiguous word.  "You
   can't beat the system."
   SYSTEM HACKER: one who hacks the system (in sense 1 only; for sense
   2 one mentions the particular program: e.g., LISP HACKER)

T [from LISP terminology for "true"] 1. Yes.  Usage: used in reply to
   a question, particularly one asked using the "-P" convention).  See
   NIL.	 2. See TIME T.

TAIL RECURSION n. See TAIL RECURSION.

TALK MODE  See COM MODE.

TASTE n. (primarily MIT-DMS) The quality in programs which tends to be
   inversely proportional to the number of features, hacks, and kluges
   programmed into it.	Also, TASTY, TASTEFUL, TASTEFULNESS.  "This
   feature comes in N tasty flavors."  Although TASTEFUL and FLAVORFUL
   are essentially synonyms, TASTE and FLAVOR are not.

TECO (tee'koe) [acronym for Text Editor and COrrector] 1. n. A text
   editor developed at MIT, and modified by just about everybody.  If
   all the dialects are included, TECO might well be the single most
   prolific editor in use.  Noted for its powerful pseudo-programming
   features and its incredibly hairy syntax.  2. v. To edit using the
   TECO editor in one of its infinite forms; sometimes used to mean
   "to edit" even when not using TECO!	Usage: rare at SAIL, where
   most people wouldn't touch TECO with a TENEX pole.
   [Historical note: DEC grabbed an ancient version of MIT TECO many
   years ago when it was still a TTY-oriented editor.  By now, TECO at
   MIT is highly display-oriented and is actually a language for
   writing editors, rather than an editor.  Meanwhile, the outside
   world's various versions of TECO remain almost the same as the MIT
   version of ten years ago.  DEC recently tried to discourage its
   use, but an underground movement of sorts kept it alive.]
   [Since this note was written I found out that DEC tried to force
   their hackers by administrative decision to use a hacked up and
   generally lobotomized version of SOS instead of TECO, and they
   revolted. - MRC]

TELNET v. To communicate with another ARPAnet host using the TELNET
   protocol.  TOPS-10 people use the word IMPCOM since that is the
   program name for them.  Sometimes abbreviated to TN.	 "I usually TN
   over to SAIL just to read the AP News."

TENSE adj. Of programs, very clever and efficient.  A tense piece of
   code often got that way because it was highly bummed, but sometimes
   it was just based on a great idea.  A comment in a clever display
   routine by Mike Kazar: "This routine is so tense it will bring
   tears to your eyes.	Much thanks to Craig Everhart and James
   Gosling for inspiring this hack attack."  A tense programmer is one
   who produces tense code.

TERPRI (tur'pree) [from the LISP 1.5 (and later, MacLISP) function to
   start a new line of output] v. To output a CRLF (q.v.).

THEORY n. Used in the general sense of idea, plan, story, or set of
   rules.  "What's the theory on fixing this TECO loss?"  "What's the
   theory on dinner tonight?"  ("Chinatown, I guess.")	"What's the
   current theory on letting losers on during the day?"	 "The theory
   behind this change is to fix the following well-known screw..."

THRASH v. To move wildly or violently, without accomplishing anything
   useful.  Swapping systems which are overloaded waste most of their
   time moving pages into and out of core (rather than performing
   useful computation), and are therefore said to thrash.

TICK n. 1. Interval of time; basic clock time on the computer.
   Typically 1/60 second.  See JIFFY.  2. In simulations, the discrete
   unit of time that passes "between" iterations of the simulation
   mechanism.  In AI applications, this amount of time is often left
   unspecified, since the only constraint of interest is that caused
   things happen after their causes.  This sort of AI simulation is
   often pejoratively referred to as "tick-tick-tick" simulation,
   especially when the issue of simultaneity of events with long,
   independent chains of causes is handwaved.

TIME T n. 1. An unspecified but usually well-understood time, often
   used in conjunction with a later time T+1.  "We'll meet on campus
   at time T or at Louie's at time T+1."  2. SINCE (OR AT) TIME T
   EQUALS MINUS INFINITY: A long time ago; for as long as anyone can
   remember; at the time that some particular frob was first designed.

TOOL v.i. To work; to study.  See HACK (def #9).

TRAP 1. n. A program interrupt, usually used specifically to refer to
   an interrupt caused by some illegal action taking place in the user
   program.  In most cases the system monitor performs some action
   related to the nature of the illegality, then returns control to
   the program.	 See UUO.  2. v. To cause a trap.  "These instructions
   trap to the monitor."  Also used transitively to indicate the cause
   of the trap.	 "The monitor traps all input/output instructions."

TTY (titty) n. Terminal of the teletype variety, characterized by a
   noisy mechanical printer, a very limited character set, and poor
   print quality.  Usage: antiquated (like the TTY's themselves).
   Sometimes used to refer to any terminal at all; sometimes used
   to refer to the particular terminal controlling a job.

TWEAK v. To change slightly, usually in reference to a value.  Also
   used synonymously with TWIDDLE.  See FROBNICATE and FUDGE FACTOR.

TWENEX n. The TOPS-20 operating system by DEC.	So named because
   TOPS-10 was a typically crufty DEC operating system for the PDP-10.
   BBN developed their own system, called TENEX (TEN EXecutive), and
   in creating TOPS-20 for the DEC-20 DEC copied TENEX and adapted it
   for the 20.	Usage: DEC people cringe when they hear TOPS-20
   referred to as "Twenex", but the term seems to be catching on
   nevertheless.  Release 3 of TOPS-20 is sufficiently different from
   release 1 that some (not all) hackers have stopped calling it
   TWENEX, though the written abbreviation "20x" is still used.

TWIDDLE n. 1. tilde (ASCII 176, "~").  Also called "squiggle",
   "sqiggle" (sic--pronounced "skig'gul"), and "twaddle", but twiddle
   is by far the most common term.  2. A small and insignificant
   change to a program.	 Usually fixes one bug and generates several
   new ones.  3. v. To change something in a small way.	 Bits, for
   example, are often twiddled.	 Twiddling a switch or knob implies
   much less sense of purpose than toggling or tweaking it; see
   FROBNICATE.

UP adj. 1. Working, in order.  "The down escalator is up."  2. BRING
   UP: v. To create a working version and start it.  "They brought up
   a down system."

USER n. A programmer who will believe anything you tell him.  One who
   asks questions.  Identified at MIT with "loser" by the spelling
   "luser".  See REAL USER.
   [Note by GLS: I don't agree with RF's definition at all.
   Basically, there are two classes of people who work with a program:
   there are implementors (hackers) and users (losers).	 The users are
   looked down on by hackers to a mild degree because they don't
   understand the full ramifications of the system in all its glory.
   (A few users who do are known as real winners.)  It is true that
   users ask questions (of necessity).	Very often they are annoying
   or downright stupid.]

UUO (you-you-oh) [short for "Un-Used Operation"] n. A DEC-10 system
   monitor call.  The term "Un-Used Operation" comes from the fact
   that, on DEC-10 systems, monitor calls are implemented as invalid
   or illegal machine instructions, which cause traps to the monitor
   (see TRAP).	The SAIL manual describing the available UUO's has a
   cover picture showing an unidentified underwater object.  See YOYO.
   [Note: DEC sales people have since decided that "Un-Used Operation"
   sounds bad, so UUO now stands for "Unimplemented User Operation".]
   Tenex and Twenex systems use the JSYS machine instruction (q.v.),
   which is halfway between a legal machine instruction and a UUO,
   since KA-10 Tenices implement it as a hardware instruction which
   can be used as an ordinary subroutine call (sort of a "pure JSR").

VANILLA adj. Ordinary flavor, standard.	 See FLAVOR.  When used of
   food, very often does not mean that the food is flavored with
   vanilla extract!  For example, "vanilla-flavored wonton soup" (or
   simply "vanilla wonton soup") means ordinary wonton soup, as
   opposed to hot and sour wonton soup.

VAXEN [from "oxen", perhaps influenced by "vixen"] n. pl. The plural
   of VAX (a DEC machine).

VIRGIN adj. Unused, in reference to an instantiation of a program.
   "Let's bring up a virgin system and see if it crashes again."
   Also, by extension, unused buffers and the like within a program.

VIRTUAL adj. 1. Common alternative to LOGICAL (q.v.), but never used
   with compass directions.  2.	 Performing the functions of.  Virtual
   memory acts like real memory but isn't.

VISIONARY n. One who hacks vision (in an AI context, such as the
   processing of visual images).

WALDO [probably taken from the story "Waldo", by Heinlein, which is
   where the term was first used to mean a mechanical adjunct to a
   human limb] Used at Harvard, particularly by Tom Cheatham and
   students, instead of FOOBAR as a meta-syntactic variable and
   general nonsense word.  See FOO, BAR, FOOBAR, QUUX.

WALL [shortened form of HELLO WALL, apparently from the phrase "up
   against a blank wall"] (WPI) interj. 1. An indication of confusion,
   usually spoken with a quizzical tone.  "Wall??"  2. A request for
   further explication.

WALLPAPER n. A file containing a listing (e.g., assembly listing) or
   transcript, esp. a file containing a transcript of all or part of a
   login session.  (The idea was that the LPT paper for such listings
   was essentially good only for wallpaper, as evidenced at SAIL where
   it was used as such to cover windows.)  Usage: not often used now,
   esp. since other systems have developed other terms for it (e.g.,
   PHOTO on TWENEX).  The term possibly originated on ITS, where the
   commands to begin and end transcript files are still :WALBEG and
   :WALEND, with default file DSK:WALL PAPER.

WATERBOTTLE SOCCER n. A deadly sport practiced mainly by Sussman's
   graduate students.  It, along with chair bowling, is the most
   evident manifestation of the "locker room atmosphere" said to
   reign in that sphere.  (Sussman doesn't approve.)  [As of 11/82,
   it's reported that the sport has given way to a new game called
   "disc-boot", and Sussman even participates occasionally.]

WEDGED [from "head wedged up ass"] adj. To be in a locked state,
   incapable of proceeding without help.  (See GRONK.)	Often refers
   to humans suffering misconceptions.	"The swapper is wedged."
   This term is sometimes used as a synonym for DEADLOCKED (q.v.).

WHAT n. The question mark character ("?").  See QUES.  Usage: rare,
   used particularly in conjunction with WOW.

WHEEL n. 1. A privilege bit that canonically allows the possessor to
   perform any operation on a timesharing system, such as read or
   write any file on the system regardless of protections, change or
   or look at any address in the running monitor, crash or reload the
   system, and kill/create jobs and user accounts.  The term was
   invented on the TENEX operating system, and carried over to
   TOPS-20, Xerox-IFS and others.  2. A person who posses a wheel bit.
   "We need to find a wheel to unwedge the hung tape drives."

WHEEL WARS [from LOTS at Stanford University] A period during which
   student wheels hack each other by attempting to log each other out
   of the system, delete each other's files, or otherwise wreak havoc,
   usually at the expense of the lesser users.

WIN [from MIT jargon] 1. v. To succeed.	 A program wins if no
   unexpected conditions arise.	 2. BIG WIN: n. Serendipity.
   Emphatic forms: MOBY WIN, SUPER WIN, HYPER-WIN (often used
   interjectively as a reply).	For some reason SUITABLE WIN is also
   common at MIT, usually in reference to a satisfactory solution to a
   problem.  See LOSE.

WINNAGE n. The situation when a lossage is corrected, or when
   something is winning.  Quite rare.  Usage: also quite rare.

WINNER 1. n. An unexpectedly good situation, program, programmer or
   person.  2. REAL WINNER: Often sarcastic, but also used as high
   praise.

WINNITUDE n. The quality of winning (as opposed to WINNAGE, which is
   the result of winning).  "That's really great!  Boy, what
   winnitude!"

WIZARD n. 1. A person who knows how a complex piece of software or
   hardware works; someone who can find and fix his bugs in an
   emergency.  Rarely used at MIT, where HACKER is the preferred term.
   2. A person who is permitted to do things forbidden to ordinary
   people, e.g., a "net wizard" on a TENEX may run programs which
   speak low-level host-imp protocol; an ADVENT wizard at SAIL may
   play Adventure during the day.

WORMHOLE n. A location in a monitor which contains the address of a
   routine, with the specific intent of making it easy to substitute a
   different routine.  The following quote comes from "Polymorphic
   Systems", vol. 2, p. 54:

   "Any type of I/O device can be substituted for the standard device
   by loading a simple driver routine for that device and installing
   its address in one of the monitor's `wormholes.'*
   ----------
   *The term `wormhole' has been used to describe a hypothetical
   astronomical situation where a black hole connects to the `other
   side' of the universe.  When this happens, information can pass
   through the wormhole, in only one direction, much as `assumptions'
   pass down the monitor's wormholes."

WOW  See EXCL.

XGP 1. n. Xerox Graphics Printer.  2. v. To print something on the
   XGP.	 "You shouldn't XGP such a large file."

XYZZY [from the Adventure game] adj. See PLUGH.

YOYO n. DEC service engineers' slang for UUO (q.v.).  Usage: rare at
   Stanford and MIT, has been found at random DEC installations.

YOYO MODE n. State in which the system is said to be when it rapidly
   alternates several times between being up and being down.

YU-SHIANG WHOLE FISH n. The character gamma (extended SAIL ASCII 11),
   which with a loop in its tail looks like a fish.  Usage: used
   primarily by people on the MIT LISP Machine.	 Tends to elicit
   incredulity from people who hear about it second-hand.

ZERO v. 1. To set to zero.  Usually said of small pieces of data, such
   as bits or words.  2. To erase; to discard all data from.  Said of
   disks and directories, where "zeroing" need not involve actually
   writing zeroes throughout the area being zeroed.

