\hypertarget{a00060}{
\section{Example applications}
\label{a00060}\index{Example applications@{Example applications}}
}


Collaboration diagram for Example applications:

\subsection{Detailed Description}
The u\-IP distribution contains a number of example applications that can be either used directory or studied when learning to develop applications for u\-IP. 



\subsection*{Files}
\begin{CompactItemize}
\item 
file \hyperlink{a00044}{memb.h}
\begin{CompactList}\small\item\em Memory block allocation routines. \item\end{CompactList}

\item 
file \hyperlink{a00043}{memb.c}
\begin{CompactList}\small\item\em Memory block allocation routines. \item\end{CompactList}

\end{CompactItemize}
\subsection*{Modules}
\begin{CompactItemize}
\item 
\hyperlink{a00079}{Web client}
\begin{CompactList}\small\item\em This example shows a HTTP client that is able to download web pages and files from web servers. \item\end{CompactList}

\item 
\hyperlink{a00080}{SMTP E-mail sender}
\begin{CompactList}\small\item\em The Simple Mail Transfer Protocol (SMTP) as defined by RFC821 is the standard way of sending and transfering e-mail on the Internet. \item\end{CompactList}

\item 
\hyperlink{a00081}{Telnet server}
\begin{CompactList}\small\item\em The u\-IP telnet server provides a command based interface to u\-IP. \item\end{CompactList}

\item 
\hyperlink{a00082}{Web server}
\begin{CompactList}\small\item\em The u\-IP web server is a very simplistic implementation of an HTTP server. \item\end{CompactList}

\end{CompactItemize}
\subsection*{Defines}
\begin{CompactItemize}
\item 
\#define \hyperlink{a00060_g8457539d6a6eaecded820f4042b8314a}{MEMB}(name, size, num)
\begin{CompactList}\small\item\em Declare a memory block. \item\end{CompactList}\end{CompactItemize}
\subsection*{Functions}
\begin{CompactItemize}
\item 
void \hyperlink{a00060_gd58a6c7e62ae59bf7a016ded12ca2910}{memb\_\-init} (struct memb\_\-blocks $\ast$m)
\begin{CompactList}\small\item\em Initialize a memory block that was declared with \hyperlink{a00060_g8457539d6a6eaecded820f4042b8314a}{MEMB()}. \item\end{CompactList}\item 
char $\ast$ \hyperlink{a00060_g73bf7c370e6ada339f102d4c9768e48c}{memb\_\-alloc} (struct memb\_\-blocks $\ast$m)
\begin{CompactList}\small\item\em Allocate a memory block from a block of memory declared with \hyperlink{a00060_g8457539d6a6eaecded820f4042b8314a}{MEMB()}. \item\end{CompactList}\item 
char \hyperlink{a00060_ga02c1627ee9488468c8cdef7fed74d91}{memb\_\-ref} (struct memb\_\-blocks $\ast$m, char $\ast$ptr)
\begin{CompactList}\small\item\em Increase the reference count for a memory chunk. \item\end{CompactList}\item 
char \hyperlink{a00060_g7174da2ea729ba661256d123f08ed272}{memb\_\-free} (struct memb\_\-blocks $\ast$m, char $\ast$ptr)
\begin{CompactList}\small\item\em Deallocate a memory block from a memory block previously declared with \hyperlink{a00060_g8457539d6a6eaecded820f4042b8314a}{MEMB()}. \item\end{CompactList}\end{CompactItemize}


\subsection{Define Documentation}
\hypertarget{a00060_g8457539d6a6eaecded820f4042b8314a}{
\index{exampleapps@{exampleapps}!MEMB@{MEMB}}
\index{MEMB@{MEMB}!exampleapps@{exampleapps}}
\subsubsection[MEMB]{\setlength{\rightskip}{0pt plus 5cm}\#define MEMB(name, size, num)}}
\label{a00060_g8457539d6a6eaecded820f4042b8314a}


{\bf Value:}

\footnotesize\begin{verbatim}static char memb_mem[(size + 1) * num]; \
        static struct memb_blocks name = {size, num, memb_mem}
\end{verbatim}\normalsize 
Declare a memory block. 

\begin{Desc}
\item[Parameters:]
\begin{description}
\item[{\em name}]The name of the memory block (later used with \hyperlink{a00060_gd58a6c7e62ae59bf7a016ded12ca2910}{memb\_\-init()}, \hyperlink{a00060_g73bf7c370e6ada339f102d4c9768e48c}{memb\_\-alloc()} and \hyperlink{a00060_g7174da2ea729ba661256d123f08ed272}{memb\_\-free()}).\item[{\em size}]The size of each memory chunk, in bytes.\item[{\em num}]The total number of memory chunks in the block. \end{description}
\end{Desc}


\subsection{Function Documentation}
\hypertarget{a00060_g73bf7c370e6ada339f102d4c9768e48c}{
\index{exampleapps@{exampleapps}!memb_alloc@{memb\_\-alloc}}
\index{memb_alloc@{memb\_\-alloc}!exampleapps@{exampleapps}}
\subsubsection[memb\_\-alloc]{\setlength{\rightskip}{0pt plus 5cm}char $\ast$ memb\_\-alloc (struct memb\_\-blocks $\ast$ {\em m})}}
\label{a00060_g73bf7c370e6ada339f102d4c9768e48c}


Allocate a memory block from a block of memory declared with \hyperlink{a00060_g8457539d6a6eaecded820f4042b8314a}{MEMB()}. 

\begin{Desc}
\item[Parameters:]
\begin{description}
\item[{\em m}]A memory block previosly declared with \hyperlink{a00060_g8457539d6a6eaecded820f4042b8314a}{MEMB()}. \end{description}
\end{Desc}
\hypertarget{a00060_g7174da2ea729ba661256d123f08ed272}{
\index{exampleapps@{exampleapps}!memb_free@{memb\_\-free}}
\index{memb_free@{memb\_\-free}!exampleapps@{exampleapps}}
\subsubsection[memb\_\-free]{\setlength{\rightskip}{0pt plus 5cm}char memb\_\-free (struct memb\_\-blocks $\ast$ {\em m}, char $\ast$ {\em ptr})}}
\label{a00060_g7174da2ea729ba661256d123f08ed272}


Deallocate a memory block from a memory block previously declared with \hyperlink{a00060_g8457539d6a6eaecded820f4042b8314a}{MEMB()}. 

\begin{Desc}
\item[Parameters:]
\begin{description}
\item[{\em m}]m A memory block previosly declared with \hyperlink{a00060_g8457539d6a6eaecded820f4042b8314a}{MEMB()}.\item[{\em ptr}]A pointer to the memory block that is to be deallocated.\end{description}
\end{Desc}
\begin{Desc}
\item[Returns:]The new reference count for the memory block (should be 0 if successfully deallocated) or -1 if the pointer \char`\"{}ptr\char`\"{} did not point to a legal memory block. \end{Desc}
\hypertarget{a00060_gd58a6c7e62ae59bf7a016ded12ca2910}{
\index{exampleapps@{exampleapps}!memb_init@{memb\_\-init}}
\index{memb_init@{memb\_\-init}!exampleapps@{exampleapps}}
\subsubsection[memb\_\-init]{\setlength{\rightskip}{0pt plus 5cm}void memb\_\-init (struct memb\_\-blocks $\ast$ {\em m})}}
\label{a00060_gd58a6c7e62ae59bf7a016ded12ca2910}


Initialize a memory block that was declared with \hyperlink{a00060_g8457539d6a6eaecded820f4042b8314a}{MEMB()}. 

\begin{Desc}
\item[Parameters:]
\begin{description}
\item[{\em m}]A memory block previosly declared with \hyperlink{a00060_g8457539d6a6eaecded820f4042b8314a}{MEMB()}. \end{description}
\end{Desc}
\hypertarget{a00060_ga02c1627ee9488468c8cdef7fed74d91}{
\index{exampleapps@{exampleapps}!memb_ref@{memb\_\-ref}}
\index{memb_ref@{memb\_\-ref}!exampleapps@{exampleapps}}
\subsubsection[memb\_\-ref]{\setlength{\rightskip}{0pt plus 5cm}char memb\_\-ref (struct memb\_\-blocks $\ast$ {\em m}, char $\ast$ {\em ptr})}}
\label{a00060_ga02c1627ee9488468c8cdef7fed74d91}


Increase the reference count for a memory chunk. 

\begin{Desc}
\item[Note:]No sanity checks are currently made.\end{Desc}
\begin{Desc}
\item[Parameters:]
\begin{description}
\item[{\em m}]m A memory block previosly declared with \hyperlink{a00060_g8457539d6a6eaecded820f4042b8314a}{MEMB()}.\item[{\em ptr}]A pointer to the memory chunk for which the reference count should be increased.\end{description}
\end{Desc}
\begin{Desc}
\item[Returns:]The new reference count. \end{Desc}
