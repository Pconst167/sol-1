\hypertarget{a00077}{
\section{CPU architecture configuration}
\label{a00077}\index{CPU architecture configuration@{CPU architecture configuration}}
}


Collaboration diagram for CPU architecture configuration:

\subsection{Detailed Description}
The CPU architecture configuration is where the endianess of the CPU on which u\-IP is to be run is specified. 

Most CPUs today are little endian, and the most notable exception are the Motorolas which are big endian. The BYTE\_\-ORDER macro should be changed to reflect the CPU architecture on which u\-IP is to be run. 

\subsection*{Defines}
\begin{CompactItemize}
\item 
\#define \hyperlink{a00077_g1771b7fb65ee640524d0052f229768c3}{BYTE\_\-ORDER}
\begin{CompactList}\small\item\em The byte order of the CPU architecture on which u\-IP is to be run. \item\end{CompactList}\end{CompactItemize}


\subsection{Define Documentation}
\hypertarget{a00077_g1771b7fb65ee640524d0052f229768c3}{
\index{uipoptcpu@{uipoptcpu}!BYTE_ORDER@{BYTE\_\-ORDER}}
\index{BYTE_ORDER@{BYTE\_\-ORDER}!uipoptcpu@{uipoptcpu}}
\subsubsection[BYTE\_\-ORDER]{\setlength{\rightskip}{0pt plus 5cm}\#define BYTE\_\-ORDER}}
\label{a00077_g1771b7fb65ee640524d0052f229768c3}


The byte order of the CPU architecture on which u\-IP is to be run. 

This option can be either BIG\_\-ENDIAN (Motorola byte order) or LITTLE\_\-ENDIAN (Intel byte order). 