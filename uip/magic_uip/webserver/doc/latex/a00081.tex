\hypertarget{a00081}{
\section{Telnet server}
\label{a00081}\index{Telnet server@{Telnet server}}
}


Collaboration diagram for Telnet server:

\subsection{Detailed Description}
The u\-IP telnet server provides a command based interface to u\-IP. 

It allows using the \char`\"{}telnet\char`\"{} application to access u\-IP, and implements the required telnet option negotiation.

The code is structured in a way which makes it possible to add commands without having to rewrite the main telnet code. The main telnet code calls two callback functions, \hyperlink{a00081_g992adc34dbe12cd28c2e5cc4c043b461}{telnetd\_\-connected()} and \hyperlink{a00081_ge639174c3eb987213a3ff0b1e138da26}{telnetd\_\-input()}, when a telnet connection has been established and when a line of text arrives on a telnet connection. These two functions can be implemented in a way which suits the particular application or environment in which the u\-IP system is intended to be run.

The u\-IP distribution contains an example telnet shell implementation that provides a basic set of commands. 

\subsection*{Files}
\begin{CompactItemize}
\item 
file \hyperlink{a00047}{telnetd.h}
\begin{CompactList}\small\item\em Header file for the telnet server. \item\end{CompactList}

\item 
file \hyperlink{a00046}{telnetd.c}
\begin{CompactList}\small\item\em Implementation of the Telnet server. \item\end{CompactList}

\item 
file \hyperlink{a00045}{telnetd-shell.c}
\begin{CompactList}\small\item\em An example telnet server shell. \item\end{CompactList}

\end{CompactItemize}
\subsection*{Data Structures}
\begin{CompactItemize}
\item 
struct \hyperlink{a00027}{telnetd\_\-state}
\begin{CompactList}\small\item\em A telnet connection structure. \item\end{CompactList}\end{CompactItemize}
\subsection*{Defines}
\begin{CompactItemize}
\item 
\hypertarget{a00081_g7764fc5e71ca9e338a6a66da54d3b308}{
\#define \hyperlink{a00081_g7764fc5e71ca9e338a6a66da54d3b308}{TELNETD\_\-LINELEN}}
\label{a00081_g7764fc5e71ca9e338a6a66da54d3b308}

\begin{CompactList}\small\item\em The maximum length of a telnet line. \item\end{CompactList}\item 
\hypertarget{a00081_gd20b6d3f9a5b992bc2de6429b37169ca}{
\#define \hyperlink{a00081_gd20b6d3f9a5b992bc2de6429b37169ca}{TELNETD\_\-NUMLINES}}
\label{a00081_gd20b6d3f9a5b992bc2de6429b37169ca}

\begin{CompactList}\small\item\em The number of output lines being buffered for all telnet connections. \item\end{CompactList}\end{CompactItemize}
\subsection*{Functions}
\begin{CompactItemize}
\item 
void \hyperlink{a00081_g992adc34dbe12cd28c2e5cc4c043b461}{telnetd\_\-connected} (struct \hyperlink{a00027}{telnetd\_\-state} $\ast$s)
\begin{CompactList}\small\item\em Callback function that is called when a telnet connection has been established. \item\end{CompactList}\item 
void \hyperlink{a00081_ge639174c3eb987213a3ff0b1e138da26}{telnetd\_\-input} (struct \hyperlink{a00027}{telnetd\_\-state} $\ast$s, char $\ast$cmd)
\begin{CompactList}\small\item\em Callback function that is called when a line of text has arrived on a telnet connection. \item\end{CompactList}\item 
void \hyperlink{a00081_g816bdc3e31e05e0979efe91a697b10ad}{telnetd\_\-close} (struct \hyperlink{a00027}{telnetd\_\-state} $\ast$s)
\begin{CompactList}\small\item\em Close a telnet session. \item\end{CompactList}\item 
void \hyperlink{a00081_g24ecabdebc734cb350bdd766c0cddf1c}{telnetd\_\-output} (struct \hyperlink{a00027}{telnetd\_\-state} $\ast$s, char $\ast$str1, char $\ast$str2)
\begin{CompactList}\small\item\em Print out a string on a telnet connection. \item\end{CompactList}\item 
void \hyperlink{a00081_g8873fd3ee516cfcca82cc4bc67f564c0}{telnetd\_\-prompt} (struct \hyperlink{a00027}{telnetd\_\-state} $\ast$s, char $\ast$str)
\begin{CompactList}\small\item\em Print a prompt on a telnet connection. \item\end{CompactList}\item 
void \hyperlink{a00081_g82ff99d50221f7c17df57dc6092ffc97}{telnetd\_\-init} (void)
\begin{CompactList}\small\item\em Initialize the telnet server. \item\end{CompactList}\end{CompactItemize}


\subsection{Function Documentation}
\hypertarget{a00081_g816bdc3e31e05e0979efe91a697b10ad}{
\index{telnetd@{telnetd}!telnetd_close@{telnetd\_\-close}}
\index{telnetd_close@{telnetd\_\-close}!telnetd@{telnetd}}
\subsubsection[telnetd\_\-close]{\setlength{\rightskip}{0pt plus 5cm}void telnetd\_\-close (struct \hyperlink{a00027}{telnetd\_\-state} $\ast$ {\em s})}}
\label{a00081_g816bdc3e31e05e0979efe91a697b10ad}


Close a telnet session. 

This function can be called from a telnet command in order to close the connection.

\begin{Desc}
\item[Parameters:]
\begin{description}
\item[{\em s}]The connection which is to be closed. \end{description}
\end{Desc}
\hypertarget{a00081_g992adc34dbe12cd28c2e5cc4c043b461}{
\index{telnetd@{telnetd}!telnetd_connected@{telnetd\_\-connected}}
\index{telnetd_connected@{telnetd\_\-connected}!telnetd@{telnetd}}
\subsubsection[telnetd\_\-connected]{\setlength{\rightskip}{0pt plus 5cm}void telnetd\_\-connected (struct \hyperlink{a00027}{telnetd\_\-state} $\ast$ {\em s})}}
\label{a00081_g992adc34dbe12cd28c2e5cc4c043b461}


Callback function that is called when a telnet connection has been established. 

\begin{Desc}
\item[Parameters:]
\begin{description}
\item[{\em s}]The telnet connection. \end{description}
\end{Desc}
\hypertarget{a00081_g82ff99d50221f7c17df57dc6092ffc97}{
\index{telnetd@{telnetd}!telnetd_init@{telnetd\_\-init}}
\index{telnetd_init@{telnetd\_\-init}!telnetd@{telnetd}}
\subsubsection[telnetd\_\-init]{\setlength{\rightskip}{0pt plus 5cm}void telnetd\_\-init (void)}}
\label{a00081_g82ff99d50221f7c17df57dc6092ffc97}


Initialize the telnet server. 

This function will perform the necessary initializations and start listening on TCP port 23. \hypertarget{a00081_ge639174c3eb987213a3ff0b1e138da26}{
\index{telnetd@{telnetd}!telnetd_input@{telnetd\_\-input}}
\index{telnetd_input@{telnetd\_\-input}!telnetd@{telnetd}}
\subsubsection[telnetd\_\-input]{\setlength{\rightskip}{0pt plus 5cm}void telnetd\_\-input (struct \hyperlink{a00027}{telnetd\_\-state} $\ast$ {\em s}, char $\ast$ {\em cmd})}}
\label{a00081_ge639174c3eb987213a3ff0b1e138da26}


Callback function that is called when a line of text has arrived on a telnet connection. 

\begin{Desc}
\item[Parameters:]
\begin{description}
\item[{\em s}]The telnet connection.\item[{\em cmd}]The line of text. \end{description}
\end{Desc}
\hypertarget{a00081_g24ecabdebc734cb350bdd766c0cddf1c}{
\index{telnetd@{telnetd}!telnetd_output@{telnetd\_\-output}}
\index{telnetd_output@{telnetd\_\-output}!telnetd@{telnetd}}
\subsubsection[telnetd\_\-output]{\setlength{\rightskip}{0pt plus 5cm}void telnetd\_\-output (struct \hyperlink{a00027}{telnetd\_\-state} $\ast$ {\em s}, char $\ast$ {\em str1}, char $\ast$ {\em str2})}}
\label{a00081_g24ecabdebc734cb350bdd766c0cddf1c}


Print out a string on a telnet connection. 

This function can be called from a telnet command parser in order to print out a string of text on the connection. The two strings given as arguments to the function will be concatenated, a carrige return and a new line character will be added, and the line is sent.

\begin{Desc}
\item[Parameters:]
\begin{description}
\item[{\em s}]The telnet connection.\item[{\em str1}]The first string.\item[{\em str2}]The second string. \end{description}
\end{Desc}
\hypertarget{a00081_g8873fd3ee516cfcca82cc4bc67f564c0}{
\index{telnetd@{telnetd}!telnetd_prompt@{telnetd\_\-prompt}}
\index{telnetd_prompt@{telnetd\_\-prompt}!telnetd@{telnetd}}
\subsubsection[telnetd\_\-prompt]{\setlength{\rightskip}{0pt plus 5cm}void telnetd\_\-prompt (struct \hyperlink{a00027}{telnetd\_\-state} $\ast$ {\em s}, char $\ast$ {\em str})}}
\label{a00081_g8873fd3ee516cfcca82cc4bc67f564c0}


Print a prompt on a telnet connection. 

This function can be called by the telnet command shell in order to print out a command prompt.

\begin{Desc}
\item[Parameters:]
\begin{description}
\item[{\em s}]A telnet connection.\item[{\em str}]The command prompt. \end{description}
\end{Desc}
