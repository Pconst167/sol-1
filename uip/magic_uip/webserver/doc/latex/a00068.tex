\hypertarget{a00068}{
\section{Serial Line IP (SLIP) protocol}
\label{a00068}\index{Serial Line IP (SLIP) protocol@{Serial Line IP (SLIP) protocol}}
}


Collaboration diagram for Serial Line IP (SLIP) protocol:

\subsection{Detailed Description}
The SLIP protocol is a very simple way to transmit IP packets over a serial line. 

It does not provide any framing or error control, and is therefore not very widely used today.

This SLIP implementation requires two functions for accessing the serial device: \hyperlink{a00068_g4e8cf1d2e3d874b7aa681a272d957ccb}{slipdev\_\-char\_\-poll()} and \hyperlink{a00068_gd78a209e9faa8a97702424b44134ced3}{slipdev\_\-char\_\-put()}. These must be implemented specifically for the system on which the SLIP protocol is to be run. 

\subsection*{Files}
\begin{CompactItemize}
\item 
file \hyperlink{a00052}{slipdev.h}
\begin{CompactList}\small\item\em SLIP header file. \item\end{CompactList}

\item 
file \hyperlink{a00051}{slipdev.c}
\begin{CompactList}\small\item\em SLIP protocol implementation. \item\end{CompactList}

\end{CompactItemize}
\subsection*{Functions}
\begin{CompactItemize}
\item 
void \hyperlink{a00068_gd78a209e9faa8a97702424b44134ced3}{slipdev\_\-char\_\-put} (\hyperlink{a00070_ge081489b4906f65a3cb18e9fbe9f8d23}{u8\_\-t} c)
\begin{CompactList}\small\item\em Put a character on the serial device. \item\end{CompactList}\item 
\hyperlink{a00070_ge081489b4906f65a3cb18e9fbe9f8d23}{u8\_\-t} \hyperlink{a00068_g4e8cf1d2e3d874b7aa681a272d957ccb}{slipdev\_\-char\_\-poll} (\hyperlink{a00070_ge081489b4906f65a3cb18e9fbe9f8d23}{u8\_\-t} $\ast$c)
\begin{CompactList}\small\item\em Poll the serial device for a character. \item\end{CompactList}\item 
void \hyperlink{a00068_g24cdb292a83c88750cdc170546038d0d}{slipdev\_\-init} (void)
\begin{CompactList}\small\item\em Initialize the SLIP module. \item\end{CompactList}\item 
void \hyperlink{a00068_gf8c1cf09a7c592ed1ea6b8595aa5f162}{slipdev\_\-send} (void)
\begin{CompactList}\small\item\em Send the packet in the uip\_\-buf and uip\_\-appdata buffers using the SLIP protocol. \item\end{CompactList}\item 
\hyperlink{a00070_gfc6499c1f28697aa3bfc2804d496fd11}{u16\_\-t} \hyperlink{a00068_g0fba24e31e1974adfdae516ddadb5ee2}{slipdev\_\-poll} (void)
\begin{CompactList}\small\item\em Poll the SLIP device for an available packet. \item\end{CompactList}\end{CompactItemize}


\subsection{Function Documentation}
\hypertarget{a00068_g4e8cf1d2e3d874b7aa681a272d957ccb}{
\index{slip@{slip}!slipdev_char_poll@{slipdev\_\-char\_\-poll}}
\index{slipdev_char_poll@{slipdev\_\-char\_\-poll}!slip@{slip}}
\subsubsection[slipdev\_\-char\_\-poll]{\setlength{\rightskip}{0pt plus 5cm}\hyperlink{a00070_ge081489b4906f65a3cb18e9fbe9f8d23}{u8\_\-t} slipdev\_\-char\_\-poll (\hyperlink{a00070_ge081489b4906f65a3cb18e9fbe9f8d23}{u8\_\-t} $\ast$ {\em c})}}
\label{a00068_g4e8cf1d2e3d874b7aa681a272d957ccb}


Poll the serial device for a character. 

This function is used by the SLIP implementation to poll the serial device for a character. It must be implemented specifically for the system on which the SLIP implementation is to be run.

The function should return immediately regardless if a character is available or not. If a character is available it should be placed at the memory location pointed to by the pointer supplied by the arguement c.

\begin{Desc}
\item[Parameters:]
\begin{description}
\item[{\em c}]A pointer to a byte that is filled in by the function with the received character, if available.\end{description}
\end{Desc}
\begin{Desc}
\item[Return values:]
\begin{description}
\item[{\em 0}]If no character is available. \item[{\em Non-zero}]If a character is available. \end{description}
\end{Desc}
\hypertarget{a00068_gd78a209e9faa8a97702424b44134ced3}{
\index{slip@{slip}!slipdev_char_put@{slipdev\_\-char\_\-put}}
\index{slipdev_char_put@{slipdev\_\-char\_\-put}!slip@{slip}}
\subsubsection[slipdev\_\-char\_\-put]{\setlength{\rightskip}{0pt plus 5cm}void slipdev\_\-char\_\-put (\hyperlink{a00070_ge081489b4906f65a3cb18e9fbe9f8d23}{u8\_\-t} {\em c})}}
\label{a00068_gd78a209e9faa8a97702424b44134ced3}


Put a character on the serial device. 

This function is used by the SLIP implementation to put a character on the serial device. It must be implemented specifically for the system on which the SLIP implementation is to be run.

\begin{Desc}
\item[Parameters:]
\begin{description}
\item[{\em c}]The character to be put on the serial device. \end{description}
\end{Desc}
\hypertarget{a00068_g24cdb292a83c88750cdc170546038d0d}{
\index{slip@{slip}!slipdev_init@{slipdev\_\-init}}
\index{slipdev_init@{slipdev\_\-init}!slip@{slip}}
\subsubsection[slipdev\_\-init]{\setlength{\rightskip}{0pt plus 5cm}void slipdev\_\-init (void)}}
\label{a00068_g24cdb292a83c88750cdc170546038d0d}


Initialize the SLIP module. 

This function does not initialize the underlying RS232 device, but only the SLIP part. \hypertarget{a00068_g0fba24e31e1974adfdae516ddadb5ee2}{
\index{slip@{slip}!slipdev_poll@{slipdev\_\-poll}}
\index{slipdev_poll@{slipdev\_\-poll}!slip@{slip}}
\subsubsection[slipdev\_\-poll]{\setlength{\rightskip}{0pt plus 5cm}\hyperlink{a00070_gfc6499c1f28697aa3bfc2804d496fd11}{u16\_\-t} slipdev\_\-poll (void)}}
\label{a00068_g0fba24e31e1974adfdae516ddadb5ee2}


Poll the SLIP device for an available packet. 

This function will poll the SLIP device to see if a packet is available. It uses a buffer in which all avaliable bytes from the RS232 interface are read into. When a full packet has been read into the buffer, the packet is copied into the uip\_\-buf buffer and the length of the packet is returned.

\begin{Desc}
\item[Returns:]The length of the packet placed in the uip\_\-buf buffer, or zero if no packet is available. \end{Desc}
\hypertarget{a00068_gf8c1cf09a7c592ed1ea6b8595aa5f162}{
\index{slip@{slip}!slipdev_send@{slipdev\_\-send}}
\index{slipdev_send@{slipdev\_\-send}!slip@{slip}}
\subsubsection[slipdev\_\-send]{\setlength{\rightskip}{0pt plus 5cm}void slipdev\_\-send (void)}}
\label{a00068_gf8c1cf09a7c592ed1ea6b8595aa5f162}


Send the packet in the uip\_\-buf and uip\_\-appdata buffers using the SLIP protocol. 

The first 40 bytes of the packet (the IP and TCP headers) are read from the uip\_\-buf buffer, and the following bytes (the application data) are read from the uip\_\-appdata buffer. 